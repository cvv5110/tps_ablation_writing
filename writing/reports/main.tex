\documentclass[12pt]{article}
\usepackage[margin=0.5in]{geometry}
\usepackage{graphicx}
\usepackage{xcolor}
\usepackage{amsmath}
\usepackage{amsfonts}
\usepackage{amssymb}
\usepackage{cleveref}

\newcommand{\hl}[1]{\colorbox{yellow}{#1}}

\newcommand{\vA}{\mathbf{A}}
\newcommand{\vB}{\mathbf{B}}
\newcommand{\vBi}{\mathbf{B}^{(i)}}
\newcommand{\vBj}{\mathbf{B}^{(j)}}
\newcommand{\vC}{\mathbf{C}}
\newcommand{\vu}{\mathbf{u}}
\newcommand{\bw}{\mathbf{w}}
\newcommand{\vw}{\mathbf{w}}
\newcommand{\bv}{\mathbf{v}}
\newcommand{\vv}{\mathbf{v}}
\newcommand{\vf}{\mathbf{f}}
\newcommand{\cN}{\mathcal{N}}
\newcommand{\vr}{\mathbf{r}}
\newcommand{\cQnet}{\mathcal{Q}_{\text{net}}}
\newcommand{\cP}{\mathcal{P}}
\newcommand{\bk}{\mathbf{k}}
\newcommand{\cQ}{\mathcal{Q}}
\newcommand{\vPhi}{\boldsymbol{\Phi}}
\newcommand{\bu}{\mathbf{u}}
\newcommand{\bepsilon}{\boldsymbol{\epsilon}}
\newcommand{\bsigma}{\boldsymbol{\sigma}}
\newcommand{\phii}{\phi^{(i)}}
\newcommand{\vk}{\mathbf{k}}
\newcommand{\phij}{\phi^{(j)}}
\newcommand{\Ti}{T^{(i)}}
\newcommand{\Tj}{T^{(j)}}
\newcommand{\ui}{u^{(i)}}
\newcommand{\uj}{u^{(j)}}
\newcommand{\vui}{\mathbf{u}^{(i)}}
\newcommand{\vuj}{\mathbf{u}^{(j)}}
\newcommand{\cV}{\mathcal{V}}
\newcommand{\barf}{\bar{\mathbf{f}}}
\newcommand{\vfBC}{\mathbf{f}_{\text{BC}}}
\newcommand{\vAi}{\mathbf{A}^{(i)}}
\newcommand{\vCi}{\mathbf{C}^{(i)}}
\newcommand{\vfQ}{\mathbf{f}_{\mathcal{Q}}}
\newcommand{\bvu}{\bar{\mathbf{u}}}
\newcommand{\vM}{\mathbf{M}}
\newcommand{\phik}{\phi_k}
\newcommand{\phil}{\phi_l}
\newcommand{\Ei}{E^{(i)}}
\newcommand{\dirichletset}{\left\{T_b\right\}}
\newcommand{\eij}{e_{ij}}
\newcommand{\jump}[1]{\left[#1\right]}
\newcommand{\average}[1]{\left\{#1\right\}}
\newcommand{\sumneighbori}{\sum_{j\in\mathcal{N}_i}}
\newcommand{\sumneighborj}{\sum_{i\in\mathcal{N}_j}}
\newcommand{\sumneighbordirichlet}{\sum_{j\in\mathcal{N}_i\cup\left\{T_b\right\}}}
\newcommand{\eiq}{e_{iq}}
\newcommand{\eiT}{e_{iT}}
\newcommand{\intEi}{\int_{E^{(i)}}}
\newcommand{\barui}{\bar{u}^{(i)}}
\newcommand{\baruj}{\bar{u}^{(j)}}
\newcommand{\Rij}{R_{ij}}
\newcommand{\cE}{\mathcal{E}}
\newcommand{\cG}{\mathcal{G}}
\newcommand{\vBiij}{\mathbf{B}^{(i)}_{ij}}
\newcommand{\vBjij}{\mathbf{B}^{(j)}_{ij}}
\newcommand{\vfi}{\mathbf{f}^{(i)}}

% d#/dt
\newcommand{\ddt}[1]{\frac{\dd #1}{\dd t}}
\newcommand{\ddtb}[1]{\frac{\dd^2#1}{\dd t^2}}
\newcommand{\ddtn}[2]{\frac{\dd^#2#1}{\dd t^#2}}
% p#/pt
\newcommand{\ppt}[1]{\frac{\partial#1}{\partial t}}
\newcommand{\pptb}[1]{\frac{\partial^2#1}{\partial t^2}}
\newcommand{\pptn}[2]{\frac{\partial^#2#1}{\partial t^#2}}

\newcommand\vn{{\mathbf{n}}}

\begin{document}

\begin{abstract}
    This work presents a physics-infused reduced-order modeling (PIROM) framework towards the design, analysis, and optimization of ablating hypersonic thermal protection systems (TPS). 
\end{abstract}

\section{Introduction}

At hypersonic speeds, aerospace vehicles experience extreme aero-thermo-dynamic environments that require specialized thermal protection systems (TPS) to shield internal sub-structures, electronics, and possibly crew members from the intense aerodynamic heating. The TPS is often composed of ablating materials -- a high-temperature capable fibrous material injected with a resin that fills the pore network and strengthens the composite~\hl{Amar2016}. The TPS design promotes the exchange of mass through thermal and chemical reactions (i.e., pyrolisis), effectively mitigating heat transfer to the sub-structures.

As a result, accurate prediction for the ablating TPS response under extreme hypersonic heating becomes fundamental to ensuring survivability, performance, and safety of hypersonic vehicles. Not only is it necessary to assess the performance of the thermal management systems, but also the shape changes of the vehicle's outer surface induced by the ablating material, and its impact on the aerodynamics, structural integrity, and controllability. Nonetheless, high-fidelity simulations of ablating TPS remains a formidable challenge both theoretically and computationally.




Unfortunately, high-fidelity simulations of ablating TPS remains a formidable challenge both theoretically and computationally.

On the theoretical side, the thermo-chemical reactions, coupled with the irregular pore network structure, translate into simplifying assumptions to reduce non-linearities, and make the resulting equations more amenable for engineering application and design analysis~\hl{x}. For instance, one of the most notable codes is the one-dimensional \hl{CMA} code that was developed by Aerotherm Corporation in the 1960s~\hl{Howard2015}. Despite its practical use in...

Another example is the CHarring Ablator Response (CHAR) ablation code, which ignores elemental decompositions of the pyrolizing gases, assumes the gases to be a mixture of perfect gases in thermal equilibrium, and assumes no reaction or condensation with the porous network~\cite{Amar2016}.

In sum, the objectives of this work are as follows:
\begin{enumerate}
    \item Formulate the PIROM for the transient thermo-ablative response of multi-layered hypersonic TPS through a systematic coarse-graining procedure based on the Mori-Zwanzig formalism.
    \item Benchmark the accuracy, generalizability, and computational efficiency of the PIROM against the RPM and the high-fidelity FEM solutions of ablating TPS, thus demonstrating the PIROM's potential to solve the ITM in complex multi-physical non-linear dynamical systems.
\end{enumerate}


\section{Modeling of Ablating Thermal Protection Systems}

This section presents the problem of modeling a non-decomposing ablating TPS subjected to extreme hypersonic heating. Two different but mathematically connected solution strategies are provided: (1) a high-fidelity full-order model (FOM) based on a finite element method (FEM), and (2) a low-fidelity reduced-physics model (RPM) based on a lumped capacitance model (LCM) and a one-dimensional surface velocity model (SVM). The FOM is computationally expensive but provides the highest fidelity, while the RPM is computationally efficeint but has low predictive fidelity; both models are amenable to high-dimensional design variables. The RPM is used in the subsequent sections for deriving the PIROM.

\subsection{Governing Equations}\label{sec_governing_equations}
The multi-physics for a non-decomposing ablating TPS involves the \textit{energy equation} which models the transient heat conduction inside the TPS, and the \textit{pseudo-elasticity equation}, which models the mesh motion due to surface recession. The governing PDEs for the ablating TPS are summarized in this section.

\subsubsection{Energy Equation}
Consider a generic domain $\Omega\subset$, $d=2$ or $3$, illustrated in Fig.~\ref{fig_general_domain}. Let $\partial\Omega = \Gamma_q\cup\Gamma_T$ and $\Gamma_q\cap\Gamma_T = \emptyset$, where a Neumann $q_b(x,t)$ boundary condition is prescribed on the $\Gamma_q$ boundary, and represents the surface exposed to the hypersonic boundary layer. The Dirichlet $T_b(x,t)$ boundary condition is prescribed on the boundary $\Gamma_T$. The TPS is divided into $N$ non-overlapping components $\left\{\Omega_i\right\}_{i=1}^{N}$, as illustrated in Fig.~\ref{fig_general_domain} for $N=2$. The $i$-th component $\Omega_i$ is associated with material properties $\left(\rho_i,c_{p,i}, \vk_i\right)$, that are assumed to be continuous within one component, and can be discontinuous across two neighboring components. 

\begin{figure}
    \centering
    \includegraphics[width=0.6\textwidth]{./figs/general_domain.png}
    \caption{General domain $\Omega$ with prescribed Neumann and Dirichlet boundary conditions on $\Gamma_q$ and $\Gamma_T$. Mesh displacement $w(x,t)$ occurs on the $\Gamma_q$ boundary.}
    \label{fig_general_domain}
\end{figure}

The transient heat conduction is described by the energy equation,
\begin{subequations}
    \begin{align}
        \rho c_p\left(\ppt{T} + \tilde{\vv}(x,t)\cdot\nabla T\right) - \nabla\cdot (\mathbf{k}\nabla T) &= 0,\ x\in\Omega \label{eqn_thermal_pde}\\
        -\mathbf{k}\nabla T\cdot \vn &= q_b(x,t),\ x\in\Gamma_q\label{eqn_thermal_bc_neumann}\\
        T(x,t) &= T_b(x,t),\ x\in\Gamma_T\label{eqn_thermal_bc_dirichlet}\\
        T(x,0) &= T_0(x),\ x\in\Omega\label{eqn_thermal_ic}
    \end{align}\label{eqn_governing_equations}
\end{subequations}
where the density $\rho$ is constant, while the heat capacity $c_p$ and thermal conductivity $\mathbf{k}\in\mathbb{R}^{d\times d}$ may depend on temperature. In the order they appear, the terms in \cref{eqn_thermal_pde} include, the unsteady energy storage, heat conduction, temperature advection due to mesh motion, and source terms due to boundary conditions. The boundary conditions for the energy equation includes Neumann \cref{eqn_thermal_bc_neumann} and Dirichlet \cref{eqn_thermal_bc_dirichlet} on $\Gamma_T$. 

An Abirtrary Lagrangian-Eulerian (ALE) description is used to account for mesh motion due to surface recession, where $\tilde{\vv}(x,t)$ is the relative velocity of the material with respect to the mesh,
\begin{equation}
    \tilde{\vv}(x,t) = \vv_s(x,t) - \vv_m(x,t)\label{eqn_relative_velocity}
\end{equation}
where $\vv_s(x,t)$ and $\vv_m(x,t)$ are the physical material velocity and mesh velocity, respectively. In this work, the physical material velocity is assumed to be zero, i.e., $\vv_s(x,t)=\vzero$, and thus the relative velocity is simply the negative of the mesh velocity, $\tilde{\vv}(x,t) = -\vv_m(x,t)$.

\subsubsection{Pseudo-Elasticity Equation}
The mesh motion is described by the steady-state pseudo-elasticity equation without body forces,
\begin{subequations}
    \begin{align}
        \nabla\cdot\vsigma(\vw) &= \vzero,\quad \forall\:t\in\cT,\:\vx\in\Omega\quad\label{eqn_elasticity_pde}\\
        \vw(\vx,t) &= \vw_q(\vx,t),\quad \forall\:t\in\cT,\: \vx\in\Gamma_q\label{eqn_displacement_heated_bc}\\
        \vw(\vx,t) &= \vzero,\quad \forall\:t\in\cT,\: \vx\notin \Gamma_q\label{eqn_displacement_unheated_bc}\\
        \vw(\vx,0) &= \vzero,\quad \forall\:\vx\in\Omega\label{eqn_displacement_initial_condition}
    \end{align}
\end{subequations}
where the stress tensor $\vsigma$ is related to the strain tensor $\vepsilon(\vw)$ through Hooke's law,
\[
    \vsigma(\vw) = \mathbb{D}:\vepsilon(\vw)
\]
where $\mathbb{D}$ is the fourth-order positive definite elasticity tensor, and ``:'' is the double contraction of the full-order tensor $\mathbb{D}$ with the second-order tensor $\vepsilon$. The elasticity tensor ordinarily possess a number of symmetries, effectively reducing the number of components that describe it~\cite{Bergel2022}. The symmetric strain tensor $\vepsilon$ measures the deformation of the mesh due to displacements $\vw(x,t)$, and is defined as,
\[
    \vepsilon(\vw) = \frac{1}{2}\left(\nabla\vw + \nabla\vw^\top\right)
\]
The ``material'' properties for the mesh are chosen to tailor the mesh deformation, and need not represent the actual material being modeled~\cite{Amar2016}.

For the pseudo-elasticity equations, the boundary conditions include prescribed displacements $\vw_q(x,t)$ on the heated boundary $\Gamma_q$ in \cref{eqn_displacement_heated_bc}, and zero displacements on the unheated boundaries in \cref{eqn_displacement_unheated_bc}. The initial condition for the mesh displacements is zero in \cref{eqn_displacement_initial_condition}. Particularly, the surface velocity due to the ablating material is a function of the surface temperature $T_q(x,t)$ for $x\in\Gamma_q$ on the heated boundary. For the $i$-th material component, the mesh velocity on the heated boundary is imposed based on the following relation,
\begin{equation}
    \hat{\vn}\cdot\vv_m(x,t) = f(T_q(x,t)),\quad x\in\Gamma_q\label{eqn_boundary_velocity}
\end{equation}
where $\hat{\vn}$ is the unit normal vector on the heated boundary $\Gamma_q$, and $f$ is a function obtained from tabulated data for the material, commonly referred to as a B' table~\cite{Amar2016}. The B' table provides a model for the recession velocity as a function of the surface temperature, and is pre-computed based on high-fidelity simulations of the ablation process for a one-dimensional slab of the material, and is independent of the TPS geometry. Provided the surface velocity, the boundary condition in \cref{eqn_boundary_displacement} for the mesh displacements are computed by integrating the surface velocity over time,
\begin{equation}
    \vw_q(x,t) = \int_{0}^{t} \mathbf{v}_m(x,\tau)d\tau\label{eqn_boundary_displacement}
\end{equation}

\subsection{Full-Order Model: Finite-Element Method}\label{sec_fom}
To obtain the full-order numerical solution, the \textit{energy equation} is spatially discretized using variational principles of the Discontinuous Galerkin (DG) method~\cite{Cohen2018}. Note that the choice of DG approach is mainly for theoretical convenience, and is exclusively performed on the energy equation, as it is the surface temperature that drives the ablation process. The equivalence between DG and FEM is noted upon their convergence. For the \textit{pseudo-elasticity equation} standard FEM is used to compute the mesh displacements based on the surface temperature provided by the DG solution of the energy equation~\cite{Bergel2022}.

Consider a conforming mesh partition domain, where each element belongs to one and only one component. Denote the collection of all $M$ elements as $\left\{E_i\right\}_{i=1}^{M}$. In an element $E_i$, its shared boundaries with another element $E_j$, Neumann BC, and Dirichlet BC are denoted as $e_{ij}$, $e_{iq}$, and $e_{iT}$, respectively. Lastly, $\left|e\right|$ denotes the length $(n_d=2)$ or area $(n_d=3)$ of a component boundary $e$.

For the $i$-th element, use a set of $P$ trial functions, such as polynomials, to represent the temperature distribution,
\begin{equation}
    T_i(x,t) = \sum_{l=1}^{P} \phi_l^i(x) u_l^i(t) \equiv \vphi_i^\top(x)\vu_i(t),\quad i=1,2,\dots,M\label{eqn_element_temperature}
\end{equation}
Without loss of generality, the trial functions are assumed to be orthogonal, so that,
\[
    \int_{E_i}\phi_l^i(x)\phi_k^i(x)d\Omega = 0,\quad l\neq k
\] 
where $\delta_{lk}$ is the Kronecker delta function. Furthermore, for simplicity, choose $\phi_1^i=1$. Thus, by orthogonality,
\[
    \int_{E_i}\phi_1^i(x)d\Omega = |E_i|,\quad \int_{E_i}\phi_k^i(x)d\Omega = 0,\quad k=2,3,\dots,P
\]
Under the choice of basis functions, $u_1^i$ is simply the average temperature of element $E_i$, denoted as $\bu_i$.

By standard variational processes, e.g.,~\cite{Cohen2018}, the element-wise governing equation is denoted as,
\begin{equation}
    \vA_i\dot{\vu}_i = \left(\vB_i + \vC_i\right)\vu_i + \sumneighbordirichlet\left(\vB^i_{ij}\vu_i + \vB^{j}_{ij}\vu_j\right) + \vf_i(t),\quad\text{for }i=1,2,\dots,M\label{eqn_element_dg}
\end{equation}
which is collected as the following ODE for the all the elements in the mesh,
\begin{equation}
    \vA(\vu)\dot{\vu} = \left[\vB(\vu) + \vC(\vu)\right]\vu + \mathbf{f}(t)\label{eqn_full_dg}
\end{equation}
where $\vu = \left[\vu_1, \vu_2, \ldots, \vu_M\right]^\top\in\mathbb{R}^{MP}$ includes all the DG variables, $\mathbf{f}\in\mathbb{R}^{MP}$ is the external forcing, and the system matrices $\vA$, $\vB$, and $\vC$ are the matrices due to heat capacity, heat conduction, and temperature advection due to mesh motion, respectively. A detailed derivation of \cref{eqn_element_dg,eqn_full_dg} and their matrices is provided in Appendix~\ref{appendix_mathematical_details}.

\subsection{Reduced-Physics Model}
The RPM for predicting the response of the ablating TPS consists of two components: (1) the \textit{lumped-capacitance model} (LCM), and (2) the \textit{surface velocity model} (SVM). The LCM is described as a first-order system of ODEs for predicting the average temperatures inside the components of the TPS, and provides a low-fidelity (under estimate) for the component's surface temperature. The SVM provides a relation between the surface temperature and the surface recession velocity based on pre-computed B' tables for the material, enabling the computation of one-dimensional surface displacements. The LCM and SVM are combined to define the RPM, providing low-fidelity estimates for the temperatures and surface recession of the ablating TPS.

\subsubsection{Lumped Capacitance Model}
A general form of the LCM is provided in this section; details regarding the derivation for the four-component TPS in Fig.~\ref{fig_four_components} are provided in Appendix~\ref{appendix_mathematical_details}. The LCM is a classical physics-based low-order model for predicting the temporal variation of average temperature in multiple interconnected components~\cite{Incropera2011}. The LCM is derived at the component level from a point of view of energy conservation, and leads to the following system of ODEs for the average temperatures on the components,
\begin{equation}
    \bar{\vA}\dot{\bar{\vu}} = \bar{\vB}\bar{\vu} + \bar{\vf}(t)\label{eqn_lcm}
\end{equation}
Where the states and inputs,
\begin{equation}
    \bvu = \left[\bar{u}_1,\bar{u}_2,\dots,\bar{u}_N\right]^\top\in\mathbb{R}^{N},\quad \bar{\vf} = \left[\bar{f}_1,\bar{f}_2,\dots,\bar{f}_N\right]^\top\in\mathbb{R}^{N}\label{eqn_lcm_states_inputs}
\end{equation}
include the average temperatures $\bar{\vu}$ and spatially-integrated inputs $\bar{\vf}$ for the $N$ components. For $i,j=1,2,\dots,N$ the $(i,j)$-th elements of the $\bar{\vA}\in\mathbb{R}^{N\times N}$, $\bar{\vB}\in\mathbb{R}^{N\times N}$, and $\bar{\vf}\in\mathbb{R}^{N}$ matrices are given by,
\begin{subequations}
    \begin{gather}
        \bar{A}_i = \begin{cases}
                \int_{\Omega_i}\rho c_p d\Omega_i, & i=j\\
                0, & i\neq j
            \end{cases},\quad \bar{B}_{ij} = \begin{cases}
            \sumneighbordirichlet\bar{B}^i_{ij}, &i=j \\
            \bar{B}^{(j)}_{ij}, & i\neq j
        \end{cases},\\ \vf_i = \begin{cases}
            |\eiq|\bar{q}_i + \frac{|\eiT|}{R_{i}}\bar{T}_i, & i=j \\ 
            0, & i\neq j
        \end{cases}
    \end{gather}\label{eqn_lcm_matrices}
\end{subequations}
where,
\begin{equation}
    \bar{q}_i = \frac{1}{|\eiq|}\int_{\eiq} q_b d\eiq,\quad \bar{T}_i = \frac{1}{|\eiT|}\int_{\eiT}T_b d\eiT,\quad \bar{B}^{i}_{ij} = -\frac{|\eij|}{R_{ij}},\quad \bar{B}_{ij}^{j} = \frac{|\eij|}{R_{ij}}\label{eqn_lcm_matrices_elements}
\end{equation}
where $R_{ij}$ is the equivalent thermal resistance between two neighboring components $\Omega_i$ and $\Omega_j$, and $R_i$ is the thermal resistance between component $\Omega_i$ and the Dirichlet boundary. The thermal resistances are computed based on the geometry and material properties of the components; details regarding their computation are provided in Appendix~\ref{appendix_mathematical_details}.

\begin{figure}
    \centering
    \subfigure[TPS Decomposition]{\includegraphics[width=0.8\textwidth]{./figs/four_components.png}}
    \subfigure[Lumped Mass Representation]{\includegraphics[width=0.8\textwidth]{./figs/lumped_mass_representation.png}}
    \caption{Partition of the TPS into three ablating and one non-ablating components with the corresponding lumped-mass representation.}
    \label{fig_four_components}
\end{figure}

\subsubsection{Surface Velocity Model}
The displacement is assumed to be \textit{one-dimensional} on the heated boundary $\Gamma_q$, i.e., the surface recedes only in the direction of the applied load, and occurs only for $\tilde{N}\leq N$ components. For example, in Fig.~\ref{fig_four_components}, the surface displacement on the heated boundary occurs only in the negative $y$-direction for the three components exposed to the hypersonic boundary layer; the fourth component is the substrate and does not ablate. Displacements along the $x$ direction is small relative to displacements in the $y$ direction, and are thus neglected. 

For the $i$-th component, the SVM considered in this work takes the form,
\begin{equation}
    \dot{\vw} = \vXi\bvu - \tvf\label{eqn_svm}
\end{equation}
where $\vXi=\text{diag}\left(\xi_1,\dots,\xi_{\tilde{N}}\right)$ and $\tvf=\left(\xi_1\bu_{0,1},\dots,\xi_{\tilde{N}}\bu_{0,\tilde{N}}\right)^\top$. The constants $\xi_i$ are small material-dependent constants, determined from the B' table, and $\bu_{0,i}$ is the constant initial temperature of the component. The SVM provides a relation between the surface temperature and the surface recession velocity based on pre-computed B' tables for the material.

\subsubsection{Coupled Reduced-Physics Model}
The LCM and SVM are combined to define the RPM for predicting the thermo-ablative response of the TPS under hypersonic boundary layers. Specifically, the RPM is defined as the LCM as in \cref{eqn_lcm}, where the \textit{geometry-dependent matrices} $\bar{\vA}$, $\bar{\vB}$, and $\bar{\vf}$ are updated at each time step based on the current surface displacements $\vw$ provided by the SVM. The RPM is formally stated as,
\begin{equation}
    \vcA(\vs)\dot{\vs} = \vcB(\vs)\vs + \vcF(t)\label{eqn_rpm}
\end{equation}
where the state $\vs=\left[\bvu,\vw\right]^\top\in\mathbb{R}^{N+\tilde{N}}$ includes the average temperatures for $N$ components and the one-dimensional surface displacements for the $\tilde{N}$ ablating components. The matrices are given as,
\begin{equation}
    \vcA = \begin{bmatrix}
        \bar{\vA}(\vw) & \vzero\\
        \vzero & \vI
    \end{bmatrix},\quad \vcB(\vs) = \begin{bmatrix}
        \bar{\vB}(\vw) & \vzero\\
        \vXi & \vzero
    \end{bmatrix},\quad \vcF(t) = \begin{bmatrix}
        \bar{\vf}(t)\\
        -\tvf
    \end{bmatrix}\label{eqn_rpm_matrices}
\end{equation}
In the matrices $\bvA$ and $\bvB$, the surface displacements $\vw$ are used to define the dimensions for the $\Omega_i$ component used in \cref{eqn_lcm_matrices,eqn_lcm_matrices_elements}, thus effectively coupling the LCM and SVM.

\subsection{Summary of Modeling Approaches}
The FOM (i.e., DG-FEM) and RPM (i.e., LCM with SVM) are two different but mathematically connected solution strategies. Specifically, the LCM in \cref{eqn_lcm} not only resembles the functional form of the DG model in \cref{eqn_full_dg}, but can be viewed as a special case of the latter, where the mesh partition is extremely coarse, and the trial and test functions are piece-wise constants. This removes all spatial variations within each component, and neglects advection effects due to mesh motion.

For example, consider the case where each component $\Omega_i$ is treated as one single element, and each element employs one constant basis function $\phi_i=1$. The element-wise DG model in \cref{eqn_element_dg} simplifies into a scalar ODE,
\begin{equation}
    \vAi = \bar{A}_i,\quad \vCi = 0, \quad\vBiij = -\sigma|\eij|,\quad \vBjij=\sigma|\eij|,\quad \vf_i = |\eiq|\bar{q}_i + \sigma|\eiT|\bar{T}_i
\end{equation}
Clearly, the LCM is a coarse zeroth-order DG model with the inverse of thermal resistance chosen as the element-wise penalty factors. Or conversely, the DG model is a refined version of LCM via \textit{hp}-adaptation.

The FOM and RPM represent two extremes in the modeling fidelity and computational cost spectrum. On one hand, the FOM is the most accurate but computationally expensive to evaluate due to the fine mesh discretizations for both the temperature and displacement fields, leading to possibly millions of state variables. On the other hand, the RPM considers only the average temperature of the material from which one-dimensional surface displacements are computed. This considerably reduces the computational cost, but sacrifices local temperature information that are critical to properly capture higher-order effects due to mesh motion and thermal gradients within each component. Thus, neither the FOM nor the RPM is an universal approach for real-world analysis, design, and optimization tasks for ablating TPS, where thousands of high-fidelity model evaluations may be necessary. This issue motivates the development of the PIROM, which can achieve the fidelity of FOM at a computational cost close to the RPM, while maintaining the generalizability to model parameters.
\section{Physics-Infused Reduced-Order Modeling}\label{sec_pirom}
The formulation of PIROM for ablating TPS starts by connecting the FOM, i.e., the DG-FEM, and the RPM, i.e., the LCM, via a coarse-graining procedure. This procedure pinpoints the missing dynamics in the LCM when compared to DG-FEM. Subsequently, the Mori-Zwanzig (MZ) formalism is employed to determine the model form for the missing dynamics in PIROM. Lastly, the data-driven identification of the missing dynamics in PIROM is presented.

\subsection{Deriving the Reduced-Physics Model via Coarse-Graining}
The subsequent coarse-graining formulation is performed on the DG-FEM in \cref{eqn_full_dg} to derive the LCM in \cref{eqn_lcm}. This process constraints the trial function space of a full-order DG model to a subset of piece-wise constants, so that the variables $\vu$, matrices $\vA$, $\vB$, and $\vC$, and forcing vector $\vf$ are all approximated using a single state associated to the average temperature. Note that the coarse-graining is exclusively performed on the thermal dynamics, as it is the surface temperature that drives the one-dimensional recession via the SVM. Hence, the coarse-graining of the mesh dynamics is not included in the following procedure.

\subsubsection{Coarse-Graining of States}
Consider a DG model as in \cref{eqn_full_dg} for M elements and an LCM as in \cref{eqn_lcm} for $N$ components; clearly $M\gg N$. Let $\cV_j = \left\{i \big| E_i\in\Omega_j\right\}$ be the indices of the elements belonging to the $j$-th component, so $E_i\in\Omega_j$ for all $i\in \cV_j$. The number of elements in the $j$-th component is $|\cV_j|$. The average temperature on $\Omega_j$ is,
\begin{equation}
    \bar{u}_j = \frac{1}{|\Omega_j|}\sum_{i\in\cV_j}\intEi \vphi^{(i)}(x)^T\vu^{(i)} d\Omega = \frac{1}{\left|\Omega_j\right|}\sum_{i\in\cV_j} \left|E_i\right|\vvarphi_i^{j\top}\vu^{(i)},\quad j=1,2,\dots,N\label{eqn_average_temperature}
\end{equation}
where $\left|\Omega_j\right|$ and $\left|E_i\right|$ denote the area $(d=2)$ or volume $(d=3)$ of component $j$ and element $i$, respectively. The orthogonal basis functions are defined as $\vvarphi_i^{j\top} = \left[1,0,\dots,0\right]^{\top}\in\mathbb{R}^{P}$.

Conversely, given the average temperatures of the $N$ components, $\bvu$, the states of an arbitrary element $E_i$ is written as,
\begin{equation}
    \vu^{(i)} = \sum_{k=1}^{N}\vvarphi_i^{k}\bar{u}_k + \deltavu^{(i)}, \quad i=1,2,\dots,M\label{eqn_element_states}
\end{equation}
where $\vvarphi_{i}^{k}=0$ if $i\notin\cV_k$, and $\deltavu^{(i)}$ represents the deviation from the average temperature and satisfies the orthogonality condition $\vvarphi_{i}^{k\top}\deltavu^{(i)}=0$ for all $k$.

Equations \cref{eqn_average_temperature,eqn_element_states} are combined and written in matrix form as,
\begin{equation}
    \bvu = \vPhiplus\vu,\quad \vu = \vPhi\vu + \deltavu
\end{equation}
where $\vPhi\in\mathbb{R}^{MP\times N}$ is a matrix of $M\times N$ blocks, with the $(i,j)$-th block as $\vvarphiij$, $\vPhiplus\in\mathbb{R}^{N\times MP}$ is the left inverse of $\vPhi$, with the $(i,j)$-th block as $\vvarphi_{i}^{j+} = \frac{|E_i|}{|\Omega_j|}\vvarphiijT$, and $\deltavu$ is the collection of deviations. By their definitions, $\vPhiplus\vPhi = \vI$ and $\vPhiplus\deltavu = \vzero$.

\subsubsection{Coarse-Graining of Dynamics}
The dependence of the matrices with respect to the displacements $\vw$ is dropped to isolate the analysis based on coarsened variables. Consider a function of states in the form of $\vM\left(\vu\right)\vg(\vu)$, where $\vg:\mathbb{R}^{MP} \to \mathbb{R}^{MP}$ is a vector-valued function, and $\vM:\mathbb{R}^{MP} \to \mathbb{R}^{p\times MP}$ is a matrix-valued function with an arbitrary dimension $p$. Define the projection matrix $\vP=\vPhi\vPhiplus$ and the projection operator $\cP$ as,
\begin{align}
    \cP\left[\vM(\vu)\vg(\vu)\right] &= \vM\left(\vP{\vu}\right)\vg\left(\vP{\vu}\right)\notag\\
    &= \vM\left(\vPhi\bvu\right)\vg\left(\vPhi\bvu\right)\label{eqn_projection_operator}
\end{align}
so that the resulting function depends only on the average temperatures $\bvu$. Correspondingly, the residual operator $\cQ = \cI - \cP$, and $\cQ\left[\vM(\vu)\vg(\vu)\right] = \vM(\vu)\vg(\vu) - \vM\left(\vPhi\bvu\right)\vg\left(\vPhi\bvu\right)$. When the function is not separable, the projection operator is simply defined as $\cP\left[\vg(\vu)\right] = \vg\left(\vP\vu\right)$.

Subsequently, the operators defined above are applied to coarse-grain the dynamics. First, write the DG-FEM in \cref{eqn_full_dg} as,
\begin{equation}
    \dot{\vu} = \vA(\vu)^{-1}\vB(\vu)\vu + \vA(\vu)^{-1}\vC(\vu)\vu + \vA(\vu)^{-1}\vf(t)\label{eqn_full_dg_rearranged}
\end{equation}
and multiply both sides by $\vPhiplus$ to obtain,
\begin{equation}
    \vPhiplus\dot{\vu} = \vPhiplus\left(\vPhi\dot{\bvu} + \delta\dot{\vu}\right) = \dot{\bvu} = \vPhiplus\vr(\vu,t)
\end{equation}
Apply the projection operator $\cP$ and the residual operator $\cQ$ to the right-hand side to obtain,
\begin{equation}
    \dot{\bvu} = \cP\left[\vPhiplus\vr(\vu,t)\right] + \cQ\left[\vPhiplus\vr(\vu,t)\right]\equiv \vr^{(1)}(\vu,t) + \vr^{(2)}(\vu,t)\label{eqn_coarse_grained_dynamics}
\end{equation}
where $\vr^{(1)}(\vu,t)$ is resolved dynamics that depends on $\bvu$ only, and $\vr^{(2)}(\vu,t)$ is the un-resolved or residual dynamics. Detailed derivations and analysis of $\vr^{(1)}(\vu,t)$ and $\vr^{(2)}(\vu,t)$ can be found in the Appendix. 

It follows from Ref.~\cite{VargasVenegas2025} that the resolved dynamics is exactly the LCM, where the advection term reduces to zero, i.e., $\bvC(\bvu) = \vzero$ as shown in the Appendix. Using the notation from \cref{eqn_lcm}, it follows that,
\begin{align}
    \vr^{(1)}(\vu,t) &= \bvA(\bvu)^{-1}\bvB(\bvu)\bvu + \bvA(\bvu)^{-1}\bvC(\bvu)\bvu + \bvA(\bvu)^{-1}\bvf(\bvu)\notag\\
    &= \bvA(\bvu)^{-1}\bvB(\bvu)\bvu + \bvA(\bvu)^{-1}\bvf(t)\label{eqn_resolved_dynamics_no_advection}
\end{align}
where the following relations hold,
\begin{subequations}
    \begin{align}
        \bvA(\bvu) &= \vW\left(\vPhiplus\vA\left(\vPhi\bvu\right)^{-1}\vPhi\right)^{-1} &\quad \bvC(\bvu) &= \vzero \\
        \bvB(\bvu) &= \vW\vPhiplus\vB\left(\vPhi\bvu\right)\vPhi &\quad \bvf(t) &= \vW\vPhiplus\vf
    \end{align}\label{eqn_coarse_grained_matrices}
\end{subequations}
where $\vW\in\mathbb{R}^{N\times N}$ is a diagonal matrix with the $i$-th element as $\left[\vW\right]_{ii} = \left|\cV_k\right|$ if $i\in\cV_k$. The examination of the second residual term $\vr^{(2)}(\vu,t)$ in \cref{eqn_coarse_grained_dynamics} is shown in the Appendix, and demonstrates that the physical sources of missing dynamics in the LCM include: the approximation of non-uniform temperature within each component as a constant, and the elimination of the advection term due to coarse-graining. In sum, the above results not only show that the LCM is a result of coarse-graining of the full-order DG-FEM, but also reveal the discrepancies between the LCM and the DG-FEM. These discrepancies propagate into the SVM, which as a result of the averaging in the LCM formulation, under-predicts the surface recession rates. In the subsequent section, the discrepancies in the LCM are corrected to formulate the PIROM.

\subsection{Physics-Infusion Via Mori-Zwanzig Formalism}
The Mori-Zwanzig (MZ) formalism is an operator-projection technique used to derive ROMs for high-dimensional dynamical systems, especially in statistical mechanics and fluid dynamics~\cite{Parish2017a,Parish2017,Duraisamy2025}. It provides an exact reformulation of a high-dimensional Markovian dynamical system, into a low-dimensional observable non-Markovian dynamical system. The proposed ROM is subsequently developed based on the approximation to the non-Markovian term in the observable dynamics. Particularly, \cref{eqn_coarse_grained_dynamics} shows that the DG-FEM dynamics can be decomposed into the resolved dynamics $\vr^{(1)}(\vu,t)$ and the orthogonal dynamics $\vr^{(2)}(\vu,t)$, in the sense of $\cP\vr^{(2)}=0$. In this case, the MZ formalism can be invoked to express the dynamics $\bvu$ in terms of $\bvu$ alone as the projected Generalized Langevin Equation (GLE)~\cite{Parish2017a,Parish2017,Duraisamy2025},
\begin{equation}
    \dot{\bvu}(t) = \vr^{(1)}(\bvu,t) + \int_{0}^{t}\tvkappa(t,s,\bvu)ds\label{eqn_gle}
\end{equation}
where the first and second terms are referred to as the Markovian and non-Markovian terms, respectively. The non-Markovian term accounts for the effects of past un-resolved states on the current resolved states via a memory kernel $\tvkappa(t,s,\bvu)$, which in practice is computationally expensive to evaluate.

\subsubsection{Markovian Reformulation}
This section details the formal derivation of the PIROM as a system of ODEs for the thermal dynamics, based on approximations to the memory kernel. Specifically, the kernel $\tvk$ is examined via a leading-order expansion, based on prior work~\cite{Yu2024coarse}; this can be viewed as an analog of zeroth-order holding in linear system theory with a sufficiently small time step. In this case, the memory kernel is approximated as,
\begin{equation}
    \tvkappa(t,s,\bvu) \approx \vr^{(1)}(\bvu,t) \cdot \nabla_{\bvu}\vr^{(2)}\left(\vPhi\bvu,t\right)
\end{equation}
Note that the terms in $\vr^{(1)}$ have a common factor $\bvA^{-1}$; this motivates the following heuristic modification of the model form in \cref{eqn_gle},
\begin{subequations}
    \begin{align}
        \dot{\bvu} &= \vr^{(1)}(\bvu,t) + \bvA^{-1}(\bvu)\int_{0}^{t}\vkappa(t,s,\bvu)ds\label{eqn_pirom_gle}\\
        \bvA(\bvu)\dot{\bvu} &= \bvB\left(\bvu\right)\bvu + \bvf(t) + \int_{0}^{t}\vkappa(t,s,\bvu)ds\label{eqn_lcm_gle}
    \end{align}
\end{subequations}
where the original kernel $\tvkappa$ is effectively normalized by $\bvA^{-1}$. Intuitively, such choice of kernel reduces its dependency on the averaged material properties, and simplifies the subsequent design of model form.

Subsequently, the hidden states are introduced to ``Markovianize'' the system \cref{eqn_gle}. In this manner, \cref{eqn_lcm_gle} is converted into a pure state-space model, with the functional form of the LCM retained; since LCM is a physics-based model, then it encodes the physical information and retains explicit parametric dependence of the problem. Consider the representation of the kernel as a finite sum of simpler functions, e.g., exponentials,
\begin{equation}
    \vkappa(t,s,\bvu) = \sum_{j=1}^{m}\cK_j(t,s,\bvu)\left[\vp_j + \vd_j(\bvu)\right]\phi_j(s,\bvu)
\end{equation}
where,
\begin{equation}
    \cK_j(t,s,\bvu) = e^{-\int_{s}^{t}\left(\lambda_j + e_j(\bvu)\right)d\tau},\quad \phi_j(s,\bvu) = \vq_j^{\top}\bvu(s) + \vg_j(\bvu)^{\top}\bvu(s) + \vr_j^{\top}\bvf(s)
\end{equation}
with suitable coefficients $\vp_j,\vd_j,\vq_j,\vg_j,\vr_j\in\mathbb{R}^{N}$ and decay rates $\lambda_j,e_j(\bvu)>0$, that need to be identified from data. 

Define the hidden states as,
\begin{equation}
    \beta_j(t) = \int_{0}^{t}\cK_j(s,\bvu)\phi_j(s,\bvu)ds
\end{equation}
then through its differentiation with respect to time,
\begin{equation}
    \dot{\beta}_j(t) = -\left[\lambda_j + e_j(\bvu)\right]\beta_j(t) + \vq_j^{\top}\bvu(t) + \vg_j(\bvu)^{\top}\bvu(t) + \vr_j^{\top}\bvf(t)
\end{equation}
and the memory term becomes,
\begin{equation}
    \int_{0}^{t}\vkappa(t,s,\bvu)ds = \sum_{j=1}^{m}\left[\vp_j + \vd_j(\bvu)\right]\beta_j(t)
\end{equation}
Then, \cref{eqn_lcm_gle} is recast as the extended Markovian system,
\begin{subequations}
    \begin{align}
        \bvA(\bvu)\dot{\bvu} &= \bvB(\bvu)\bvu + \left[\vP + \vD(\bvu)\right]\vbeta + \bvf(t)\\
        \dot{\vbeta} &= \left[\vQ + \vG(\bvu)\right]\bvu + \left[\vE(\bvu) - \vLambda\right]\vbeta + \vR\bvf(t)
    \end{align}\label{eqn_extended_markovian_system}
\end{subequations}
where the data-driven operators associated to the hidden dynamics are collected as,
\begin{subequations}
    \begin{align}
        \vLambda &= \text{diag}\left(\lambda_1,\lambda_2,\dots,\lambda_m\right)\in\mathbb{R}^{m\times m},\quad & \vP &= \left[\vp_1,\vp_2,\dots,\vp_m\right]\in\mathbb{R}^{N\times m}\\
        \vD(\bvu) &= \left[\vd_1(\bvu),\vd_2(\bvu),\dots,\vd_m(\bvu)\right]\in\mathbb{R}^{N\times m},\quad & \vQ &= \left[\vq_1,\vq_2,\dots,\vq_m\right]\in\mathbb{R}^{m\times N}\\
        \vG(\bvu) &= \left[\vg_1(\bvu),\vg_2(\bvu),\dots,\vg_m(\bvu)\right]\in\mathbb{R}^{m\times N},\quad & \vR &= \left[\vr_1,\vr_2,\dots,\vr_m\right]\in\mathbb{R}^{m\times N}\\
        \vE(\bvu) &= \text{diag}\left(e_1(\bvu),e_2(\bvu),\dots,e_m(\bvu)\right)\in\mathbb{R}^{m\times m}
    \end{align}
\end{subequations}
The form of the temperature-dependent matrices $\vD(\bvu)$, $\vG(\bvu)$, and $\vE(\bvu)$ is provided in the next section. Note that since the hidden states $\vbeta$ serve as the memory, their initial conditions are set to zero, i.e., $\vbeta(t_0) = \vzero$, no memory at the beginning. The physics-infused model in \cref{eqn_extended_markovian_system} retains the structure of the LCM, while the hidden states account for missing physics through corrections to the stiffness and advection matrices, as well as the forcing term.

\subsubsection{Coupled Physics-Infused Model}
The next step involves coupling the physics-infused model in \cref{eqn_extended_markovian_system} with the SVM in \cref{eqn_svm} to form the PIROM for ablating TPS. To this end, define the observables as the surface temperature $\vz_u\in\mathbb{R}^{\tN}$ and displacements $\vz_w\in\mathbb{R}^{\tN}$ for $\tN\leq N$ ablating components to define the observable vector as $\vz=\left[\vz_u,\vz_w\right]^\top\in\mathbb{R}^{n_z}$ with $n_z=2\tN$ as the total number of observables.

Collect the RPM and hidden states into a single state vector $\vy = \left[\bvu,\vw,\vbeta\right]^\top\in\mathbb{R}^{n_y}$, where $n_y = N + \tN + m$, and define a data-driven operator $\vM\in\mathbb{R}^{n_z\times n_{y}}$ to define the PIROM's observable as,
\begin{equation}
    \vz = \vM\vy\label{eqn_pirom_observable_mapping}
\end{equation}
where,
\begin{equation}
    \vM = \begin{bmatrix}
        \vM_{\bu} & \vzero & \vM_{\beta}\\
        \vzero & \vI & \vzero
    \end{bmatrix}
\end{equation}
includes the matrices $\vM_{\bu}\in\mathbb{R}^{\tN\times N}$ and $\vM_{\beta}\in\mathbb{R}^{\tN\times m}$, which computes the surface temperature observable from the RPM states and hidden states, respectively. The PIROM is coupled to the SVM in \cref{eqn_svm} by leveraging \cref{eqn_pirom_observable_mapping} to compute the surface recession velocity. Thus, the PIROM is formally stated as,
\begin{subequations}
    \begin{align}
        \vcA\dot{\vy} &= \left[\vcB + \vcC\right]\vy + \vcF(t)\\
        \vz &= \vM\vy
    \end{align}\label{eqn_pirom_general}
\end{subequations}
where,
\begin{subequations}
    \begin{gather}
        \vcA = \begin{bmatrix}
            \bvA & \vO & \vO\\
            \vO & \vI & \vO\\
            \vO & \vO & \vI
        \end{bmatrix}\in\mathbb{R}^{n_y\times n_y},\quad \vcB = \begin{bmatrix}
            \bvB & \vO & \vP\\
            \vXi\vM_u & \vO & \vXi\vM_{\beta}\\
            \vQ & \vO & -\vLambda
        \end{bmatrix}\in\mathbb{R}^{n_y\times n_y},\\
        \vcC = \begin{bmatrix}
            \vO & \vO & \vD(\bvu)\\
            \vO & \vO & \vO\\
            \vG(\bvu) & \vO & \vE(\bvu)
        \end{bmatrix}\in\mathbb{R}^{n_y\times n_y},\quad
        \vcF = \begin{bmatrix}
            \bvf(t) \\
            -\tvf\\
            \vR\bvf(t)
        \end{bmatrix}\in\mathbb{R}^{n_y}
    \end{gather}
\end{subequations}

The learnable parameters in the PIROM are collected as,
\begin{equation}
    \vTheta = \left\{\vP,\vQ,\vR,\vD(\bvu),\vG(\bvu),\vE(\bvu),\vM_{u},\vM_{\beta}\right\},\in\mathbb{R}^{n_{\theta}}
\end{equation}
Particularly, the matrices $\vP,\vLambda,\vQ,\vR$ are constants that need to be identified from data, and account for the effects of coarse-graining on the stiffness and forcing matrices. The matrices $\vD(\bvu),\vE(\bvu),\vG(\bvu)$ are state-dependent matrices, and account for the effects of coarse-graining on the advection matrix due to mesh motion. Leveraging the DG-FEM formula for the advection matrix in \cref{eqn_advection_matrix_element_kl} in the Appendix, and noting that the ablating velocity in \cref{eqn_boundary_velocity} imposes the boundary condition for the mesh motion, the state-dependent matrices for the $i$-th component are written as,
\begin{equation}
    \vD(\bvu) \approx \dot{\vw}(\bvu)\odot_{\text{r}}\vD,\quad \vG(\bvu) \approx \vG\odot_{\text{r}}\dot{\vw}(\bvu), \quad \vE(\bvu) \approx \dot{\vW}(\bvu)\odot\vE\label{eqn_advection_matrix_corrections}
\end{equation}
where $\dot{\vw}(\bvu)$ is the SVM based on the observable temperature $\bvu$, $\odot_{\text{r}}$ is the row-wise multiplication, and $\dot{\vW}$ is the concatenation of $\dot{\vw}$ for $\tilde{m}$ times, where $\tilde{m}$ corresponds to the number of hidden states per component, i.e., $m = N\tilde{m}$.

The PIROM in \cref{eqn_pirom_general} incorporates explicit information on the material properties, boundary conditions, and surface recession, and is designed to generalize across parametric variations in these inputs. Moreover, the hidden dynamics in \cref{eqn_extended_markovian_system} are interpretable, as these retain the functional form of the DG-FEM in \cref{eqn_full_dg}. The next step is focused on identifying the unknown data-driven parameters $\vTheta$ characterizing the hidden dynamics.

\subsection{Learning the Hidden Dynamics}
Learning of the PIROM is achieved through a gradient-based neural-ODE-like approach~\cite{Chen2019}. For ease of presentation, consider the compact form of the PIROM in \cref{eqn_pirom_general},
\begin{equation}
    \vcD\left(\dot{\vy},\vy,\vxi,\vcF;\vTheta\right) = \vzero\label{eqn_pirom_compact}
\end{equation}
where $\vxi$ defines the model parameters, i.e., material properties and B' tables, while $\vcF$ represents the forcing terms, i.e., the boundary conditions.

Consider a dataset of $N_s$ high-fidelity \textit{surface temperature} observable trajectories $\vz_{\text{HF}}$, sampled at $p$ time instances $\{t_k\}_{k=0}^{p-1}$, for different parameter settings $\{\vxi^{(l)}\}_{l=1}^{N_s}$ and forcing functions $\{\vcF^{(l)}(t)\}_{l=1}^{N_s}$. The dataset is expressed as,
\begin{equation}
    \cD = \left\{\left(t_k,\vz^{(l)}_{\text{HF}}(t_k),\vxi^{(l)},\vcF^{(l)}(t_k)\right)\right\}_{l=1}^{N_s},\quad k=0,1,\dots,p-1
\end{equation}
In this work, the dataset contains only surface temperature observables -- all high-fidelity information regarding the surface displacements \textit{are assumed to be unavailable during learning}.

The learning problem is formulated as the following differentially-constrained problem,
\begin{subequations}
    \begin{align}
        \min_{\vTheta} &\quad\cJ\left(\vTheta;\cD\right) = \sum_{l=1}^{N_s}\int_{t_0}^{t_f}\ell\left(\vz_u^{(l)},\vz_{\text{HF}}^{(l)}\right)dt\\
        \text{s.t.}\quad \vzero &= \vcD\left(\dot{\vy}^{(l)},\vy^{(l)},\vxi^{(l)},\vcF^{(l)};\vTheta\right)
    \end{align}
\end{subequations}
for $l=1,2,\dots,N_s$, the objective is to minimize the discrepancy between the high-fidelity and PIROM predictions for the $l$-th trajectory with $\ell\left(\vz_u^{(l)},\vz_{\text{HF}}^{(l)}\right) = \left\|\vz_u^{(l)} - \vz^{(l)}_{\text{HF}}\right\|_2^2$.

The gradient-based optimization loop is based on the adjoint variable $\vlambda$, governed by the adjoint differential equation,
\begin{subequations}
    \begin{align}
        \frac{\partial\ell}{\partial\vy} + \vlambda^\top\frac{\partial\vcD}{\partial\vy} - \frac{d}{dt}\left(\vlambda^\top\frac{\partial\vcD}{\partial\dot{\vy}}\right) &= \vzero\\
        \vlambda(t_f)^\top\frac{\partial\vcD}{\partial\dot{\vy}(t_f)} &=\vzero
    \end{align}\label{eqn_adjoint_equation}
\end{subequations}
Once $\vlambda$ is solved, the gradient is computed as,
\begin{equation}
    \nabla_{\vTheta}\cJ = \frac{1}{N_s}\sum_{l=1}^{N_s}\int_{t_0}^{t_f}\left(\frac{\partial\ell}{\partial\vTheta} + \left(\vlambda^{(l)}\right)^\top\frac{\partial\vcD}{\partial\vTheta}\right)dt
\end{equation}

\hl{Discussion on TSA?}

\appendix

\section{Numerical Implementation}\label{app_implementation}

\subsection{Full-Order Model}

\subsection{Reduced-Physics Model}

This section outlines the derivation of the RPM for a generic TPS geometry. The RPM is formulated for $N$ interconnected one-dimensional components, and a three-component example is provided to demonstrate 

This section outlines the approach to derive the RPM for a series of $N$ interconnected one-dimensional components inside a TPS. Consider the splitting of the TPS into a series of one-dimensional strands, where the temperature distribution over the $n$-th strand is governed by,
\begin{equation}
    \rho c_p\frac{\partial T^{(n)}}{\partial t} - \rho c_p v^{(n)}(x,t)\frac{\partial T^{(n)}}{\partial x} - \frac{\partial}{\partial x}\left(k\frac{\partial T^{(n)}}{\partial x}\right) - \cQ^{(n)}_{\text{net}}(x,t) = 0
\end{equation}



where the temperature distribution of the $n$-th strand is related to the temperature distribution of the $(n-1)$-th and $(n+1)$-th strands via the volumetric energy source term,
\begin{equation}
    \cQ^{(n,n+1)}(x,t) = \frac{T^{(n)}(x,t) - T^{(n+1)}(x,t)}{R_{(n,)}}
\end{equation}
which is approximated using the \hl{x} relation from 

strand is coupled to 

each one-dimensional strand is coupled to the temperature distribution from its neighboring strand 

Let $\phi^{(e)}_i(x)$ with $i=1,2$ be two linear shape defined over the element $e_i=[x_{i},x_{i+1}]$ with length $h_e=x_{i+1} - x_i$,
\begin{equation}
    \phi^{(e)}_1(x) = \left\{\begin{matrix}
        \frac{x_{i+1} - x}{h_e},\quad x\in[x_i,x_{i+1}]\\
        0, \quad \text{otherwise}
    \end{matrix}\right.,\quad\phi^{(e)}_2(x) = \left\{\begin{matrix}
        \frac{x - x_i}{h_e},\quad x\in[x_i,x_{i+1}]\\
        0, \quad \text{otherwise}
    \end{matrix}\right.
\end{equation}
Letting,
\[
    x(\xi) = \frac{1-\xi}{2}x_i + \frac{1 + \xi}{2}x_{i+1}
\]
for $\xi\in[-1,1]$,
\begin{equation}
    \hat{\phi}_1^{(e)}(\xi) = \frac{1 - \xi}{2},\quad \hat{\phi}_2^{(e)}(\xi) = \frac{1 + \xi}{2}
\end{equation}

Multiply through by the test function,
\begin{equation}
    \int_{\Omega}\left[\rho c_p\frac{\partial T}{\partial t} - \rho c_p v(x,t)\frac{\partial T}{\partial x} - \frac{\partial}{\partial x}\left(k\frac{\partial T}{\partial x}\right) - \cQ(x,t)\right]\phi_i(x)dx = 0
\end{equation}

\begin{equation}
    \int_{\Omega}\rho c_p\frac{\partial T}{\partial t}\phi_i(x)dx - \int_{\Omega}\rho c_p v(x,t)\frac{\partial T}{\partial x}\phi_i(x)dx + \int_{\Omega} k\frac{\partial T}{\partial x}\frac{\partial\phi_i(x)}{\partial x} dx = k\frac{\partial T}{\partial x}\phi_i(x)\Bigg|_{\partial\Omega} + \int_{\Omega}\phi_j(x)\cQ(x,t)dx
\end{equation}
Perform the finite-element approximation,
\begin{equation}
    T(x,t)\approx\sum_jT_j(t)\phi_j(x)
\end{equation}
and define the matrix elements,
\begin{align}
    A_{ij} &= \int_{\Omega}\rho c_p\phi_i(x)\phi_j(x)dx\\
    C_{ij}(t) &= \int_{\Omega}\rho c_p v(x,t) \frac{\partial\phi_j}{\partial x}\phi_i(x)dx\\
    B_{ij} &= \int_{\Omega}k\frac{\partial\phi_i(x)}{\partial x}\frac{\partial\phi_j(x)}{\partial x}dx\\
    f_i(t) &= k\frac{\partial T}{\partial x}\phi_i(x)\Bigg|_{\partial\Omega} + \int_{\Omega}\phi_j(x)\cQ(x,t)dx
\end{align}
The time-dependent finite-dimensional ODE system for nodal temperatures $\mathbf{T}(t)$, including the ALE-induced advection effect from mesh motion, is given as,
\begin{equation}
    \mathbf{A}\frac{d\mathbf{T}}{dt} + \left(\mathbf{B} - \mathbf{C}(t)\right)\mathbf{T} = \mathbf{f}(t)
\end{equation}

The element-level expressions for the mass, stiffness, advection, and forcing vectors are given as,
\begin{align}
    M^{(e)}_{mn} &= \int_{x_i}^{x_{i+1}}\rho c_p\phi_m(x)\phi_n(x)dx = \rho c_p \frac{h_e}{6}\begin{pmatrix}
        2 & 1 \\ 1 & 2
    \end{pmatrix}\\
    K^{(e)}_{mn} &= \int_{x_i}^{x_{i+1}} k \frac{\partial \phi_m}{\partial x}\frac{\partial \phi_n}{\partial x} dx = \frac{k}{h_e}\begin{pmatrix}
        1 & -1 \\ -1 & 1
    \end{pmatrix}\\
    C^{(e)}_{mn}(t) &= \int_{x_i}^{x_{i+1}}\rho c_p v(x,t) \frac{\partial \phi_n(x)}{\partial x}\phi_m(x) dx\\
    f^{(e)}_1(t) &= \left(q(t),0\right)^T
\end{align}

\subsection{Three-Component Example}

\begin{subequations}
    \begin{align}
        \rho c_p \left(\frac{\partial T^{(1)}}{\partial t} - v^{(1)}(x,t)\frac{\partial T^{(1)}}{\partial x}\right) - \cQ^{(1)}_{\text{net}}(x,t) &= 0\\
        \rho c_p \left(\frac{\partial T^{(2)}}{\partial t} - v^{(2)}(x,t)\frac{\partial T^{(2)}}{\partial x}\right) - \cQ^{(2)}_{\text{net}}(x,t) &= 0\\
        \rho c_p \left(\frac{\partial T^{(3)}}{\partial t} - v^{(3)}(x,t)\frac{\partial T^{(3)}}{\partial x}\right) - \cQ^{(3)}_{\text{net}}(x,t) &= 0\\
    \end{align}
\end{subequations}
where,
\begin{subequations}
    \begin{align}
        \cQ^{(1)}_{\text{net}}(x,t) &=\\
        \cQ^{(2)}_{\text{net}}(x,t) &=\\
    \end{align}
\end{subequations}

\end{document}
