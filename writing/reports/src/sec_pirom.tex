\section{Physics-Infused Reduced-Order Modeling}\label{sec_pirom}
The formulation of PIROM for ablating TPS starts by connecting the FOM, i.e., DG-FEM, and the RPM, i.e., the LCM, via a coarse-graining procedure. This procedure pinpoints the missing dynamics in the LCM when compared to DG-FEM. Subsequently, the Mori-Zwanzig (MZ) formalism is employed to determine the model form for the missing dynamics in PIROM. Lastly, the data-driven identification of the missing dynanmics in PIROM is presented.

\subsection{Deriving the Reduced-Physics Model via Coarse-Graining}
The LCM is derived from a full-order DG on a fine mesh via coarse graining. This process constraints the trial function space of a full-order DG model to a subset of piece-wise constants, so that the variables $\vu$, matrices $\vA$, $\vB$, and $\vC$, and forcing vector $\vf$ are all approximated using a single state associated to the average temperature. The details of the projection are described next.

\subsubsection{Coarse-Graining of States}

Consider a DG model as in \cref{eqn_full_dg} for M elements and an LCM as in \cref{eqn_lcm} for $N$ components; clearly $M\gg N$. Let $\cV_j = \left\{i \big| E_i\in\Omega_j\right\}$ be the indices of the elements belonging to the $j$-th component, so $E_i\in\Omega_j$ for all $i\in \cV_j$. The number of elements in the $j$-th component is $|\cV_j|$. The average temperature on $\Omega_j$ is,
\begin{equation}
    \bar{u}_j = \frac{1}{|\Omega_j|}\sum_{i\in\cV_j}\intEi \vphi^{(i)}(x)^T\vu^{(i)} d\Omega = \frac{1}{\left|\Omega_j\right|}\sum_{i\in\cV_j} \left|E_i\right|\vvarphi_i^{j\top}\vu^{(i)},\quad j=1,2,\dots,N\label{eqn_average_temperature}
\end{equation}
where $\left|\Omega_j\right|$ and $\left|E_i\right|$ denote the area $(d=2)$ or volume $(d=3)$ of component $j$ and element $i$, respectively. The orthogonal basis functions are defined as $\vvarphi_i^{j\top} = \left[1,0,\dots,0\right]^{\top}\in\mathbb{R}^{P}$.

Conversely, given the average temperatures of the $N$ components, $\bvu$, the states of an arbitrary element $E_i$ is written as,
\begin{equation}
    \vu^{(i)} = \sum_{k=1}^{N}\vvarphi_i^{k}\bar{u}_k + \deltavu^{(i)}, \quad i=1,2,\dots,M\label{eqn_element_states}
\end{equation}
where $\vvarphi_{i}^{k}=0$ if $i\notin\cV_k$, and $\deltavu^{(i)}$ represents the deviation from the average temperature and satisfies the orthogonality condition $\vvarphi_{i}^{k\top}\deltavu^{(i)}=0$ for all $k$.

Equations \cref{eqn_average_temperature,eqn_element_states} are combined and written in matrix form as,
\begin{equation}
    \bvu = \vPhiplus\vu,\quad \vu = \vPhi\vu + \deltavu
\end{equation}
where $\vPhi\in\mathbb{R}^{MP\times N}$ is a matrix of $M\times N$ blocks, with the $(i,j)$-th block as $\vvarphiij$, $\vPhiplus\in\mathbb{R}^{N\times MP}$ is the left inverse of $\vPhi$, with the $(i,j)$-th block as $\vvarphi_{i}^{j+} = \frac{|E_i|}{|\Omega_j|}\vvarphiijT$, and $\deltavu$ is the collection of deviations. By their definitions, $\vPhiplus\vPhi = \vI$ and $\vPhiplus\deltavu = \vzero$.

\subsubsection{Coarse-Graining of Dynamics}
Next, consider a function of states in the form of $\vM\left(\vu\right)\vg(\vu)$, where $\vg:\mathbb{R}^{MP} \to \mathbb{R}^{MP}$ is a vector-valued function, and $\vM:\mathbb{R}^{MP} \to \mathbb{R}^{p\times MP}$ is a matrix-valued function with an arbitrary dimension $p$. Define the projection matrix $\vP=\vPhi\vPhiplus$ and the projection operator $\cP$ as,
\begin{align}
    \cP\left[\vM(\vu)\vg(\vu)\right] &= \vM\left(\vP{\vu}\right)\vg\left(\vP{\vu}\right)\notag\\
    &= \vM\left(\vPhi\bvu\right)\vg\left(\vPhi\bvu\right)\label{eqn_projection_operator}
\end{align}
so that the resulting function depends only on the average temperatures $\bvu$. Correspondingly, the residual operator $\cQ = \cI - \cP$, and $\cQ\left[\vM(\vu)\vg(\vu)\right] = \vM(\vu)\vg(\vu) - \vM\left(\vPhi\bvu\right)\vg\left(\vPhi\bvu\right)$. When the function is not separable, the projection operator is simply defined as $\cP\left[\vg(\vu)\right] = \vg\left(\vP\vu\right)$.

Subsequently, the operators defined above are applied to coarse-grain the dynamics. First, write the DG-FEM in \cref{eqn_full_dg} as,
\begin{equation}
    \dot{\vu} = \vA(\vu)^{-1}\vB(\vu)\vu + \vA(\vu)^{-1}\vC(\vu)\vu + \vA(\vu)^{-1}\vf(t)\label{eqn_full_dg_rearranged}
\end{equation}
and multiply both sides by $\vPhiplus$ to obtain,
\begin{equation}
    \vPhiplus\dot{\vu} = \vPhiplus\left(\vPhi\dot{\bvu} + \delta\dot{\vu}\right) = \dot{\bvu} = \vPhiplus\vr(\vu,t)
\end{equation}
Apply the projection operator $\cP$ and the residual operator $\cQ$ to the right-hand side to obtain,
\begin{equation}
    \dot{\bvu} = \cP\left[\vPhiplus\vr(\vu,t)\right] + \cQ\left[\vPhiplus\vr(\vu,t)\right]\equiv \vr^{(1)}(\vu,t) + \vr^{(2)}(\vu,t)\label{eqn_coarse_grained_dynamics}
\end{equation}
where $\vr^{(1)}(\vu,t)$ is resolved dynamics that depends on $\bvu$ only, and $\vr^{(2)}(\vu,t)$ is the un-resolved or residual dynamics. Detailed derivations and analysis of $\vr^{(1)}(\vu,t)$ and $\vr^{(2)}(\vu,t)$ can be found in the Appendix. 

It follows from Ref.\hl{x} that the resolved dynamics is exactly the LCM, where the advection term reduces to zero, i.e., $\bvC(\bvu) = \vzero$ as shown in the Appendix. Using the notation from \cref{eqn_lcm}, it follows that,
\begin{align}
    \vr^{(1)}(\vu,t) &= \bvA(\bvu)^{-1}\bvB(\bvu)\bvu + \bvA(\bvu)^{-1}\bvC(\bvu)\bvu + \bvA(\bvu)^{-1}\bvf(\bvu)\notag\\
    &= \bvA(\bvu)^{-1}\bvB(\bvu)\bvu + \bvA(\bvu)^{-1}\bvf(t)\label{eqn_resolved_dynamics_no_advection}
\end{align}
where the following relations hold,
\begin{subequations}
    \begin{align}
        \bvA(\bvu) &= \vW\left(\vPhiplus\vA\left(\vPhi\bvu\right)^{-1}\vPhi\right)^{-1} &\quad \bvC(\bvu) &= \vzero \\
        \bvB(\bvu) &= \vW\vPhiplus\vB\left(\vPhi\bvu\right)\vPhi &\quad \bvf(t) &= \vW\vPhiplus\vf
    \end{align}\label{eqn_coarse_grained_matrices}
\end{subequations}
where $\vW\in\mathbb{R}^{N\times N}$ is a diagonal matrix with the $i$-th element as $\left[\vW\right]_{ii} = \left|\cV_k\right|$ if $i\in\cV_k$. As shown in the Appendix, the examination of the second residual term $\vr^{(2)}(\vu,t)$ in \cref{eqn_coarse_grained_dynamics} reveals the physical sources of missing dynamics in the LCM: the approximation of non-uniform temperature within each component as a constant, and the elimination of the advection term due to coarse-graining.

In sum, the above results not only show that the LCM is a result of coarse-graining of the full-order DG-FEM, but also reveal the discrepancies between the LCM and the DG-FEM. In the subsequent section, the discrepancies will be corrected to produce the proposed PIROM.

\subsection{Formulation of Reduced-Order Model}

The Mori-Zwanzig (MZ) formalism is an operator-projection technique used to derive ROMs for high-dimensional dynamical systems, especially in statistical mechanics and fluid dynamics \hl{Parish,Duraisamy}. It provides an exact reformulation of the full-order dynamics in terms of a subset of resolved variables. The proposed ROM is subsequently developed based on such reformulation. Equation \cref{eqn_coarse_grained_dynamics} shows that the DG-FEM dynamics can be decomposed into the resolved dynamics $\vr^{(1)}(\vu,t)$ and the orthogonal dynamics $\vr^{(2)}(\vu,t)$, in the sense of $\cP\vr^{(2)}=0$. In this case, the MZ formalism can be invoked to express the dynamics $\bvu$ in terms of $\bvu$ alone as the projected Generalized Langevin Equation (GLE) \hl{Parish,Duraisamy},
\begin{equation}
    \dot{\bvu}(t) = \vr^{(1)}(\bvu,t) + \int_{0}^{t}\tvkappa(t,s,\bvu)ds\label{eqn_gle}
\end{equation}
where the first term is Markovian, and the integral term is referred to as the memory. The integral term is non-Markovian, accounting for impact of past resolved states on the current states through their interactions with the un-resolved states.

Next, to further inform the subsequent derivation of the ROM, the kernel $\tvk$ is examined via a leading-order expansion, based on prior work \hl{x}; this can be viewed as an analog of zeroth-order holding in linear system theory with a sufficiently small time step. In this case, the memory kernel is approximated as,
\begin{equation}
    \tvkappa(t,s,\bvu) \approx \vr^{(1)}(\bvu,t) \cdot \nabla_{\bvu}\vr^{(2)}\left(\vPhi\bvu,t\right)
\end{equation}
Note that the terms in $\vr^{(1)}$ have a common factor $\bvA^{-1}$; this motivates the following heuristic modification of the model form in \cref{eqn_gle},
\begin{subequations}
    \begin{align}
        \dot{\bvu} &= \vr^{(1)}(\bvu,t) + \bvA^{-1}(\bvu)\int_{0}^{t}\vkappa(t,s,\bvu)ds\label{eqn_pirom_gle}\\
        \bvA(\bvu)\dot{\bvu} &= \bvB\left(\bvu\right)\bvu + \bvf(t) + \int_{0}^{t}\vkappa(t,s,\bvu)ds\label{eqn_lcm_gle}
    \end{align}
\end{subequations}
where the original kernel $\tvkappa$ is effectively normalized by $\bvA^{-1}$. Intuitively, such choice of kernel reduces its dependency on the averaged material properties, and simplifies the subsequent design of model form.

Subsequently, the hidden states are introduced to ``Markovianize'' the system \cref{eqn_gle}. In this manner, \cref{eqn_lcm_gle} is converted into a pure state-space model, with the functional form of the LCM retained; since LCM is a physics-based model, then it encodes the physical information and retains explicit parametric dependence of the problem. Consider the representation of the kernel as a finite sum of simpler functions, e.g., exponentials,
\begin{equation}
    \vkappa(t,s,\bvu) = \sum_{j=1}^{m}\cK_j(t,s,\bvu)\left[\vp_j + \vd_j(\bvu)\right]\phi_j(s,\bvu)
\end{equation}
where,
\begin{equation}
    \cK_j(t,s,\bvu) = e^{-\int_{s}^{t}\left(\lambda_j + e_j(\bvu)\right)d\tau},\quad \phi_j(s,\bvu) = \vq_j^{\top}\bvu(s) + \vg_j(\bvu)^{\top}\bvu(s) + \vr_j^{\top}\bvf(s)
\end{equation}
with suitable coefficients $\vp_j,\vd_j,\vq_j,\vg_j,\vr_j\in\mathbb{R}^{N}$ and decay rates $\lambda_j,e_j(\bvu)>0$, that need to be identified from data. 

Define the hidden states as,
\begin{equation}
    \beta_j(t) = \int_{0}^{t}\cK_j(s,\bvu)\phi_j(s,\bvu)ds
\end{equation}
then through its differentiation with respect to time,
\begin{equation}
    \dot{\beta}_j(t) = -\left[\lambda_j + e_j(\bvu)\right]\beta_j(t) + \vq_j^{\top}\bvu(t) + \vg_j(\bvu)^{\top}\bvu(t) + \vr_j^{\top}\bvf(t)
\end{equation}
and the memory term becomes,
\begin{equation}
    \int_{0}^{t}\vkappa(t,s,\bvu)ds = \sum_{j=1}^{m}\left[\vp_j + \vd_j(\bvu)\right]\beta_j(t)
\end{equation}
Then, \cref{eqn_lcm_gle} is recast as the extended Markovian system,
\begin{subequations}
    \begin{align}
        \bvA(\bvu)\dot{\bvu} &= \bvB(\bvu)\bvu + \left[\vP + \vD(\bvu)\right]\vbeta + \bvf(t)\\
        \dot{\vbeta} &= \left[-\vLambda + \vE(\bvu)\right]\vbeta + \left[\vQ + \vG(\bvu)\right]\bvu + \vR\bvf(t)
    \end{align}\label{eqn_extended_markovian_system}
\end{subequations}
where,
\begin{subequations}
    \begin{align}
        \vLambda &= \text{diag}\left(\lambda_1,\lambda_2,\dots,\lambda_m\right)\in\mathbb{R}^{N\times m},\quad & \vP &= \left[\vp_1,\vp_2,\dots,\vp_m\right]\in\mathbb{R}^{N\times m}\\
        \vD(\bvu) &= \left[\vd_1(\bvu),\vd_2(\bvu),\dots,\vd_m(\bvu)\right]\in\mathbb{R}^{N\times m},\quad & \vQ &= \left[\vq_1,\vq_2,\dots,\vq_m\right]\in\mathbb{R}^{N\times m}\\
        \vG(\bvu) &= \left[\vg_1(\bvu),\vg_2(\bvu),\dots,\vg_m(\bvu)\right]\in\mathbb{R}^{N\times m},\quad & \vR &= \left[\vr_1,\vr_2,\dots,\vr_m\right]\in\mathbb{R}^{N\times m}\\
        \vE(\bvu) &= \text{diag}\left(e_1(\bvu),e_2(\bvu),\dots,e_m(\bvu)\right)\in\mathbb{R}^{m\times m}&&
    \end{align}
\end{subequations}
Since the hidden states $\vbeta$ serve as the memory, their initial conditions are set to zero, i.e., $\vbeta(t_0) = \vzero$, no memory at the beginning. The physics-infused model in \cref{eqn_extended_markovian_system} retains the structure of the LCM, while the hidden states account for missing physics through corrections to the stiffness and advection matrices, as well as the forcing term.

Lastly, denote the collection of resolved and hidden states as $\vy = \left[\bvu,\vbeta\right]^T\in\mathbb{R}^{n_y}$ with $n_y = N+m$, then the proposed PIROM is summarized as,
\begin{subequations}\label{eqn_pirom_general}
    \begin{align}
        \tilde{\vA}\dot{\vy} &= \left[\tilde{\vB} +\tilde{\vC}\right]\vy + \vH\bvf(t)\\
        \vz &= \vM\vy
    \end{align}
\end{subequations}
where,
\begin{subequations}
    \begin{gather}
        \tilde{\vA} = \begin{bmatrix}
            \bvA(\bvu) & \vO\\
            \vO & \vI
        \end{bmatrix}\in\mathbb{R}^{n_y\times n_y},\quad \tilde{\vB} = \begin{bmatrix}
            \bvB(\bvu) & \vP\\
            \vQ & -\vLambda
        \end{bmatrix}\in\mathbb{R}^{n_y\times n_y},\quad \tilde{\vC} = \begin{bmatrix}
            \vzero & \vD(\bvu)\\
            \vG(\bvu) & \vE(\bvu)
        \end{bmatrix}\in\mathbb{R}^{n_y\times n_y},\\
        \vH = \begin{bmatrix}
            \vI \\
            \vR
        \end{bmatrix}\in\mathbb{R}^{n_y\times N},\quad  \vM\in\mathbb{R}^{n_z\times n_y}
    \end{gather}
\end{subequations}

In \cref{eqn_pirom_general}, the terms $\bvA$, $\bvB$, and $\bvf$ are the LCM terms. The collection of matrices,
\begin{equation}
    \vTheta = \left\{\vP,\vD(\bvu),\vG(\bvu),\vLambda,\vE(\bvu),\vQ,\vG(\bvu),\vR,\vM\right\},\in\mathbb{R}^{n_{\theta}}
\end{equation}
are learnable parameters to capture the memory effects. Particularly, the matrices $\vP,\vLambda,\vQ,\vR$ are constants that need to be identified from data, and account for the effects of coarse-graining on the stiffness and forcing matrices. The matrices $\vD(\bvu),\vE(\bvu),\vG(\bvu)$ are state-dependent matrices, and account for the effects of coarse-graining on the advection matrix. Leveraging the DG-FEM formula for the advection matrix in \cref{eqn_advection_matrix_element_kl} in the Appendix, and noting that the mesh displacements are functions of the ablating velocity as in \cref{eqn_boundary_velocity}, the state-dependent matrices for the $i$-th component are written as,
\begin{equation}
    \vD^{(i)}(\bvu) \approx f^{(i)}(\bu^{(i)})\vD^{(i)}, \quad \vE^{(i)}(\bvu) \approx f^{(i)}(\bu^{(i)})\vE^{(i)}, \quad \vG^{(i)}(\bvu) \approx f^{(i)}(\bu^{(i)})\vG^{(i)}\label{eqn_advection_matrix_corrections}
\end{equation}
where $f^{(i)}(\bu^{(i)})$ is the surface recession velocity function in \cref{eqn_lcm_boundary_velocity} for the $i$-th component, and $\vD^{(i)},\vE^{(i)},\vG^{(i)}$ are constant matrices to be identified from data. In \cref{eqn_pirom_general}, $\vM$ is a fully-populated matrix that extracts the observables, i.e., the surface temperatures, from the PIROM states $\vy$. The PIROM incorporates explicit information on the temperature-dependent material properties through the LCM matrices, as well as the surface recession velocity function through \cref{eqn_advection_matrix_corrections}. The next step is focused on identifying the unknown parameters $\vTheta$ characterizing the hidden dynamics.

\subsection{Learning the Hidden Dynamics from Data}

The learning of the PIROM is achieved through a neural-ODE like approach~\hl{Chen2018}. For ease of presentation, consider the following compact form of the PIROM in \cref{eqn_pirom_general},
\begin{equation}
    \vcF\left(\dot{\vy},\vy;\vxi,\vTheta\right) = \vzero\label{eqn_pirom_compact}
\end{equation}
where $\vxi$ defines the parametrization of the problem, i.e., operating conditions, such as the BC's, as well as the material properties. Consider a dataset of $N_s$ high-fidelity trajectories of observables over a time interval $[t_0,t_f]$,
\begin{equation}
    \cD = \left\{\left(t_k,\vz^{(l)}_{\text{HF}}(t_k),\vxi^{(l)}\right)\right\}_{l=1}^{N_s},\quad k=0,1,\dots,K
\end{equation}

The learning problem is formulated as the following differentially-constrained problem,
\begin{subequations}
    \begin{align}
        \min_{\vTheta} &\quad \frac{1}{N_s}\sum_{l=1}^{N_s}\frac{1}{K}\sum_{k=0}^{K}\left\|\vz^{(l)}_{\text{HF}}(t_k) - \vM\vy^{(l)}(t_k)\right\|_2^2 \\
        \text{s.t.} &\quad \vcF\left(\dot{\vy}^{(l)},\vy^{(l)};\vxi^{(l)},\vTheta\right) = \vzero,\quad t\in[t_0,t_f],\quad l=1,2,\dots,N_s\\
        &\quad \vy^{(l)}(t_0) = \vy_0(\vxi^{(l)}),\quad l=1,2,\dots,N_s
    \end{align}
\end{subequations}


