\section{Introduction}

At hypersonic speeds, aerospace vehicles experience extreme aero-thermal environments that requires specialized thermal protection systems (TPS) to shield internal sub-structures, electronics, and possibly crew members from the intense aerodynamic heating. The TPS is often composed of ablating materials -- a high-temperature capable fibrous material injected with a resin that fills the pore network and strengthens the composite~\hl{Amar2016}. The TPS design promotes the exchange of mass through thermal and chemical reactions (i.e., pyrolisis), effectively mitigating heat transfer to the sub-structures.

As a result, accurate prediction for the ablating TPS response under extreme hypersonic heating becomes fundamental to ensuring survivability, performance, and safety of hypersonic vehicles. Not only is it necessary to assess the performance of the thermal management systems, but also the shape changes of the vehicle's outer surface induced by the ablating material, and its impact on the aerodynamics, structural integrity, and controllability. Unfortunately, high-fidelity simulations of ablating TPS remains a formidable challenge both theoretically and computationally.

On the theoretical side, the thermo-chemical reactions, coupled with the irregular pore network structure, translate into simplifying assumptions to reduce non-linearities, and make the resulting equations more amenable for engineering application and design analysis~\hl{x}. For instance, one of the most notable codes is the one-dimensional \hl{CMA} code that was developed by Aerotherm Corporation in the 1960s~\hl{Howard2015}. Despite its practical use in...

Another example is the CHarring Ablator Response (CHAR) ablation code, which ignores elemental decompositions of the pyrolizing gases, assumes the gases to be a mixture of perfect gases in thermal equilibrium, and assumes no reaction or condensation with the porous network~\cite{Amar2016}.



\hl{theoretically:}

\hl{computationally:}
