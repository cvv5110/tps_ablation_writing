\section{Modeling of Ablating Thermal Protection Systems}

This section presents the ablation problem for a non-decomposing TPS as a parametrized system of non-linear PDEs. These non-linear PDEs govern the energy of heat conduction and the pseudo-elastic material deformation of the mesh motion. Two different but mathematically-connected numerical solution strategies are provided: (1) a high-fidelity full-order model (FOM) based on a discontinuous Galerkin FEM, and (2) a thermo-elastic RPM based on a one-dimensional approximation to the energy and pseudo-elasticity equations.

\subsection{Governing Equations}\label{sec_governing_equations}

Consider a generic domain $\Omega\subset$, $d=2$ or $3$, illustrated in Fig.~\ref{fig_general_domain}. A heat flux $q_b(x,t)$ is prescribed on the boundary $\Gamma_q$ (i.e., Neumann boundary condition), and the temperature $T_b(x,t)$ is prescribed on boundary $\Gamma_T$ (i.e., Dirichlet boundary condition), where $\Gamma_q\cup\Gamma_T = \partial\Omega$ and $\Gamma_q\cap\Gamma_T = \emptyset$. The ablation occurs only on the heated boundary $\Gamma_q$, and its effects are included into the energy equation using an Arbitrary Lagrangian-Eulerian (ALE) description. The ALE assumes that the displacement $\vw(x,t)\in\mathbb{R}^d$ of the computational mesh moves with velocity $\vv(x,t)$ that is different to the material velocity, which is fixed to zero in this work.

\begin{figure}
    \centering
    \includegraphics[width=0.6\textwidth]{./figs/general_domain.png}
    \caption{General domain $\Omega$ with prescribed heat flux $q_b(x,t)$ and temperature $T_b(x,t)$ on the boundaries $\Gamma_q$ and $\Gamma_T$, respectively. The mesh moves with a velocity $\mathbf{v}(x,t)$, while the material velocity is $\mathbf{w}(x,t)$.\hl{draw mesh next to arbitrary domain with moving boundaries.}}
    \label{fig_general_domain}
\end{figure}

The transient heat conduction is described by the energy equation,
\begin{subequations}
    \begin{align}
        \rho c_p\left(\ppt{T} - \mathbf{v}(x,t)\cdot\nabla T\right) - \nabla\cdot (\mathbf{k}\nabla T) &= \cQ(x,t),\ x\in\Omega \label{eqn_thermal_pde}\\
        -\mathbf{k}\nabla T\cdot \vn &= q_b(x,t),\ x\in\Gamma_q\label{eqn_thermal_bc_neumann}\\
        T(x,t) &= T_b(x,t),\ x\in\Gamma_T\label{eqn_thermal_bc_dirichlet}\\
        T(x,0) &= T_0(x),\ x\in\Omega\label{eqn_thermal_ic}
    \end{align}\label{eqn_governing_equations}
\end{subequations}
while the mesh motion is described by the pseudo-elasticity equation,
\begin{subequations}
    \begin{align}
        \nabla\cdot\boldsymbol{\sigma}(\mathbf{w}) &= 0\label{eqn_elasticity_pde}\\
        \vw(x,t) &= \vw_q(x,t),\quad x\in\Gamma_q\label{eqn_displacement_heated_bc}\\
        \vw(x,t) &= 0,\quad x\notin \Gamma_q\label{eqn_displacement_unheated_bc}\\
        \vw(x,0) &= \boldsymbol{0}\label{eqn_displacement_initial_condition}
    \end{align}
\end{subequations}

The density $\rho$, heat capacity $c_p$, and thermal conductivity $\mathbf{k}\in\mathbb{R}^{n_d\times n_d}$ are assumed to be constant with respect to temperature in this work. The terms in \cref{eqn_thermal_pde}, in the order they appear, correspond to the unsteady energy storage, heat conduction, temperature advection due to mesh motion, and the heat source terms. 

The elasticity equation \cref{eqn_elasticity_pde} states that the divergence of the stress tensor $\boldsymbol{\sigma}(\mathbf{w})$ is zero. The stress tensor is related to the strain tensor $\bepsilon(\bw)$ through Hooke's law,
\[
    \boldsymbol{\sigma}(\bw) = \mathbb{D}:\boldsymbol{\epsilon}(\bw)
\]
where $\mathbb{D}$ is the constitutive operator, ``:'' is the double contraction of tensors, and $\bepsilon$ is the symmetric strain tensor given by,
\[
    \bepsilon(\bw) = \frac{1}{2}\left(\nabla\bw + \nabla\bw^T\right)
\]
For instance, an isotropic material assumption results in,
\[
    \bsigma = \lambda\left(\nabla\cdot\bw\right) \mathbf{I} + 2\mu\bepsilon(\bw)
\]
where $\lambda$ and $\mu$ are Lame constants that are arbitrarily selected to model the mesh motion. The ``material'' properties $\lambda$ and $\mu$ can be chosen to tailor the mesh deformation and need not represent the actual material being modeled~\hl{Amar2016}. 

The boundary conditions for the energy equation includes a heated surface (\cref{eqn_thermal_bc_neumann}) and a constant-temperature surface (\cref{eqn_thermal_bc_dirichlet}). The boundary conditions for the pseudo-elasticity equation are a function of the surface temperature $T_q(x,t)$ for $x\in\Gamma_q$ using a B' table. The B' table....
\begin{equation}
    \bw_q(x,t) = \int_{0}^{t} \mathbf{v}(x,\tau)d\tau = \int_{0}^{t}\mathbf{f}\left(T_q(x,\tau)\right)d\tau\label{eqn_boundary_displacement}
\end{equation}


\subsection{Full-Order Model: Finite-Element Method}\label{sec_fom}

To obtain the full-order numerical solution, the governing equation is spatially discretized using the variational principle from Discontinuous Galerking (DG) to result in a high-dimensional system of ODEs for the time-varying nodal data. The full-order TPS ablation simulations are computed using standard FEM instead, and the equivalence between DG and standard FEM is noted upon their convergence.

Consider a conforming mesh partition domain, where each element belongs to one and only one component. Denote the collection of all $M$ elements as $\left\{E_i\right\}_{i=1}^{M}$. In an element $E_i$, its shared boundaries with another element $E_j$, Neumann BC, and Dirichlet BC are denoted as $e_{ij}$, $e_{iq}$, and $e_{iT}$, respectively. Lastly, $\left|e\right|$ denotes the length $(n_d=2)$ or area $(n_d=3)$ of a component boundary $e$.

For the $i$-th element, use a set of $n^{(i)}$ trial functions, such as polynomials, to represent the temperature distribution,
\begin{equation}
    T^{(i)}(x,t) = \sum_{i=1}^{n^{(i)}} \phi_i^{(i)}(x)u_i^{(i)} \equiv \boldsymbol{\phi}^{(i)}(x)^T\vu^{(i)}(t)\label{eqn_element_temperature}
\end{equation}


By standard variational processes, e.g., \hl{Cohen2018}, the full governing equation is denoted as,
\begin{equation}
    \vA(\vu)\dot{\vu} = \left(\vB(\vu) + \vC(t)\right)\vu + \mathbf{f}(t)\label{eqn_energy_fem}
\end{equation}
where $\vu = \left(\vu^{(1)}, \vu^{(2)}, \ldots, \vu^{(M)}\right)^T\in\mathbb{R}^{MP}$ includes all the DG variables, $\mathbf{f}\in\mathbb{R}^{MP}$ is the external forcing, and the system matrices $\vA$, $\vB$, and $\vC$ are due to heat capacity, heat conduction, and temperature advection, respectively. The detailed formulation of the DG model is provided in Appendix~\hl{DG-FEM}.

\subsection{Reduced-Physics Model: Extended Lumped-Capacitance Model}

This section presents the main results regarding the derivation of the ELCM for ablating materials, and the main details are provided in Appendix~\hl{x}. The RPM is based on an model-order adaptation to the temperature over each element, incorporating spatial information of the temperature on the original LCM, which enables computation of the surface temperature. With the surface temperature, the recession rate is predicted.

\subsubsection{Finite-Element Method and Component Interactions}

The arbitrary domain in Fig.~\ref{fig_general_domain} is partitioned into $N$ components $\left\{\Omega^{(i)}\right\}_{i=1}^{N}$, each with $\left\{E^{(i)}_j\right\}_{j=1}^{n^{(i)}}$ finite elements, establishing the resolution for the temperature and mesh displacement fields over each component. Figure~\hl{x} shows a partition of the TPS into $N=M=3$, where the top and bottom are subject to Neumann and adiabatic boundary conditions, respectively. 

\begin{figure}
    \centering
    \includegraphics[width=0.8\textwidth]{./figs/three_components.png}
    \caption{Partition of the TPS into three one-dimensional components.}
    \label{fig_domain_partition}
\end{figure}

A first-order FEM scheme is adopted for each component, which results in a block-diagonal system of ODEs for the nodal temperature values of the components,
\begin{equation}
    \vA\left(\bvu\right)\dot{\bvu} = \left(\vB + \vC(t)\right)\bvu + \vf\left(\bvu,t\right)\label{eqn_rpm}
\end{equation}
where the block matrices are defined as,
\begin{subequations}
\begin{align}
\begin{aligned}
    \vA_{ij} &= \begin{cases}
        \vA^{(i)}(\bvu^{(i)}), & i=j \\
        0, & i\neq j
    \end{cases}\\
    \vB_{ij} &= \begin{cases}
        \vB^{(i)}(\bvu^{(i)}), & i=j \\
        0, & i\neq j
    \end{cases}
\end{aligned}
&\qquad
\begin{aligned}
    \vC_{ij}(t) &= \begin{cases}
        \vC^{(i)}(t), & i=j \\
        0, & i\neq j
    \end{cases}\\
    \mathbf{f}_i(\bvu,t) &= \begin{cases}
        \mathbf{f}^{(i)}_{\text{BC}}(t)+\mathbf{f}^{(i)}_{\cQ}(\bvu,t), & i=j \\
        \boldsymbol{0}, & i\neq j
    \end{cases}
\end{aligned}\label{eqn_rpm_matrices}
\end{align}
\end{subequations}
where $\vfBC$ and $\vfQ$ are the boundary and component-level energy sources. 


\subsubsection{Coarse Graining}

Consider a DG model as in ~\cref{eqn_energy_fem} for $M$ components and $N$ elements; clearly $N\gg N$. Let $\cV_j=\left\{i | E^{(i)}\in\Omega^{(j)}\right\}$ be the indices of the elements belonging to the $j$-th component, so $E^{(i)}\in\Omega^{(j)}$ for $i\in\cV_j$; number of elements in $\Omega^{(j)}$ is denoted as $\left|\cV_j\right|$.



The ablation on the $i$-th component is modeled using a one-dimensional approximation to the temperature and mesh-motion equations in \cref{eqn_governing_equations}, and are given by,
\begin{subequations}
    \begin{align}
        \rho c_p\left(\frac{\partial T^{(i)}}{\partial t} - v^{(i)}(x,t)\frac{\partial T^{(i)}}{\partial x}\right) - \frac{\partial}{\partial x}\left(k\frac{\partial T^{(i)}}{\partial x}\right) - \cQ^{(i)}_{\text{net}}(x,t) &= 0\label{eqn_thermal_1d}\\
        \frac{\partial}{\partial x}\left(\frac{\partial u^{(i)}}{\partial x}\right) &= 0\label{eqn_elasticity_1d}
    \end{align}\label{eqn_rpm}
\end{subequations}
with boundary conditions for the energy equation,
\begin{subequations}
    \begin{align}
        -k\frac{\partial T^{(i)}}{\partial x}\Bigg|_{x=0} &= q^{(i)}_b(t)\\
        -k\frac{\partial T^{(i)}}{\partial x}\Bigg|_{x=\ell} &= 0
    \end{align}
\end{subequations}
and for the elasticity equation,
\begin{subequations}
    \begin{align}
        u^{(i)}(0,t) &= \int_{t_0}^{t}v^{(i)}(\tau)d\tau = \int_{0}^{t} f(T^{(i)}_w(\tau))d\tau\\
        u^{(i)}(\ell,t) &= 0
    \end{align}
\end{subequations}
where $v^{(i)}(t)$ is the surface receding velocity due to ablation, which is a function of the surface temperature as in \cref{eqn_boundary_displacement}. The surface velocity is computed from a cubic spline interpolate to a B' look-up table...



\subsubsection{Thermal Solver}

The FEM implementation details are supplied in Appendix~\hl{x}. For the $n$-th component, the result of the FEM discretization is a system of ODEs for the nodal temperatures, coupled to the neighboring component $n+1$ through the energy volumetric source term,
\begin{equation}
    \mathbf{A}^{(i)}\frac{d\mathbf{T}^{(i)}}{dt} + \left(\mathbf{B}^{(i)} - \mathbf{C}^{(i)}(t)\right)\mathbf{T}^{(i)} = \mathbf{f}^{(i)}(t)
\end{equation}
where,
\begin{itemize}
    \item $\mathbf{A}^{(i)}\in\mathbb{R}^{M\times M}$ is the mass matrix,
    \item $\mathbf{B}^{(i)}\in\mathbb{R}^{M\times M}$ is the stiffness matrix,
    \item $\mathbf{C}^{(i)}(t)\in\mathbb{R}^{M\times M}$ is the advection matrix,,
    \item $\mathbf{T}^{(i)}\in\mathbb{R}^{M}$ is the vector of nodal temperatures, and
    \item $\mathbf{f}^{(i)}(t)\in\mathbb{R}^{M}$ is the input vector, which includes the Neumann boundary conditions and the net volumetric energy source term $\cQ^{(i)}_{\text{net}}$.
\end{itemize}
where $M$ is the number of nodes in the one-dimensional mesh for the $i$-th component.

\subsubsection{Pseudo-Elastic Solver}

Note that \cref{eqn_elasticity_1d} is steady. Under the assumption that the mesh deformation is quasi-steady, it can be applied at each time step within an ablation simulation. For instance, a known value of the wall temperature $T_w(t)$ specifies a Dirichlet boundary condition for the displacement, and the resulting nodal displacements within the ablator are determined from \cref{eqn_elasticity_pde}.

Along the one-dimensional domain, the PDE in \cref{eqn_elasticity_pde} simplifies to,
\begin{equation}
    \frac{\partial^2 u^{(i)}}{\partial x^2} = 0
\end{equation}
which has the analytical solution,
\begin{equation}
    u^{(i)}(x,t) = a(t)x + b(t)
\end{equation}
Imposing the boundary conditions leads to,
\begin{equation}
    u^{(i)}(x,t) = u^{(i)}(0,t)\left(\frac{x_1^{(i)} - x}{h^{(i)}}\right)
\end{equation}
The mesh velocity is the time derivative of the displacement,
\begin{equation}
    v^{(i)}(x,t) = \frac{\partial u^{(i)}(x,t)}{\partial t} = v^{(i)}(t)\left(\frac{x_1^{(i)} - x}{h^{(i)}}\right)
\end{equation}

\subsubsection{Coupling Scheme}

\subsubsection{Reduced-Physics Ablation Simulation}







