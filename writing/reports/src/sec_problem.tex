\section{Modeling of Ablating Thermal Protection Systems}

This section presents the ablation problem for a non-decomposing TPS as a parametrized system of non-linear PDEs. These non-linear PDEs govern the energy of heat conduction and the pseudo-elastic material deformation of the mesh motion. Two different but mathematically-connected numerical solution strategies are provided: (1) a high-fidelity full-order model (FOM) based on a discontinuous Galerkin FEM, and (2) a thermo-elastic RPM based on a one-dimensional approximation to the energy and pseudo-elasticity equations.

\subsection{Governing Equations}

Consider a generic domain $\Omega\subset$, $d=2$ or $3$, illustrated in Fig.~\ref{fig_general_domain}. A heat flux $q_b(x,t)$ is prescribed on the boundary $\Gamma_q$ (i.e., Neumann boundary condition), and the temperature $T_b(x,t)$ is prescribed on boundary $\Gamma_T$ (i.e., Dirichlet boundary condition), where $\Gamma_q\cup\Gamma_T = \partial\Omega$ and $\Gamma_q\cap\Gamma_T = \emptyset$. The ablation occurs only on the heated boundary $\Gamma_q$, and its effects are included into the energy equation using an Arbitrary Lagrangian-Eulerian (ALE) description. The ALE assumes the computational mesh moves with a velocity $\mathbf{v}(x,t)$, which is different to the material velocity, which is set to zero in this work.

\begin{figure}
    \centering
    \includegraphics[width=0.6\textwidth]{./figs/general_domain.png}
    \caption{General domain $\Omega$ with prescribed heat flux $q_b(x,t)$ and temperature $T_b(x,t)$ on the boundaries $\Gamma_q$ and $\Gamma_T$, respectively. The mesh moves with a velocity $\mathbf{v}(x,t)$, while the material velocity is $\mathbf{w}(x,t)$.\hl{draw mesh next to arbitrary domain with moving boundaries.}}
    \label{fig_general_domain}
\end{figure}

The transient heat conduction is coupled with the pseudo-elastic mesh motion equation as the following non-linear system of PDEs,
\begin{subequations}
    \begin{align}
        \rho c_p\ppt{T} - \nabla\cdot (\mathbf{k}(T) \nabla T) - \rho c_p\mathbf{v}(x,t)\cdot\nabla T - \cQ(x,t) &= 0,\ x\in\Omega \label{eqn_thermal_pde}\\
        \nabla\cdot\boldsymbol{\sigma}(\mathbf{w}) &= 0\label{eqn_elasticity_pde}
    \end{align}\label{eqn_governing_equations}
\end{subequations}
The density $\rho$ is constant, while the heat capacity $c_p=c_p(T)$ and thermal conductivity $\mathbf{k}=\mathbf{k}(T)\in\mathbb{R}^{d\times d}$ may vary with temperature. The boundary and initial conditions for the heat equation are,
\begin{subequations}
    \begin{align}
        -\mathbf{k}\nabla T\cdot \vn &= q_b(x,t),\ x\in\Gamma_q\label{eqn_thermal_bc_neumann}\\
        T(x,t) &= T_b(x,t),\ x\in\Gamma_T\label{eqn_thermal_bc_dirichlet}\\
        T(x,0) &= T_0(x),\ x\in\Omega\label{eqn_thermal_ic}
    \end{align}\label{eqn_thermal_bc}
\end{subequations}
while for the pseudo-elastic mesh motion, the boundary and initial conditions are,
\begin{subequations}
    \begin{align}
        \bw(x,t) &= \mathbf{w}_q(x,t),\ x\in\Gamma_q\\
        \bw(x,t) &= 0,\ x\notin\Gamma_q\\
        \bw(x,0) &= 0,\ x\in\Omega
    \end{align}\label{eqn_elasticity_bc}
\end{subequations}

The thermo-elastic equations~\cref{eqn_governing_equations}, along with the boundary conditions in \cref{eqn_thermal_bc,eqn_elasticity_bc} are described next. The terms in \cref{eqn_thermal_pde}, in the order they appear, correspond to the unsteady energy storage, heat conduction, temperature advection due to mesh motion, and heat source terms. The \cref{eqn_elasticity_pde} implies that the divergence of the stress tensor $\boldsymbol{\sigma}(\mathbf{w})$ is zero. The stress tensor is related to the strain tensor $\bepsilon(\bw)$ through Hooke's law,
\[
    \boldsymbol{\sigma}(\bw) = \mathbb{D}:\boldsymbol{\epsilon}(\bw)
\]
where $\mathbb{D}$ is the constitutive operator, ``:'' is the double contraction of tensors, and $\bepsilon$ is the symmetric strain tensor given by,
\[
    \bepsilon(\bw) = \frac{1}{2}\left(\nabla\bw + \nabla\bw^T\right)
\]
For instance, an isotropic material assumption results in,
\[
    \bsigma = \lambda\left(\nabla\cdot\bw\right) \mathbf{I} + 2\mu\bepsilon(\bw)
\]
where $\lambda$ and $\mu$ are Lame constants that are arbitrarily selected to model the mesh motion. The ``material'' properties $\lambda$ and $\mu$ can be chosen to tailor the mesh deformation and need not represent the actual material being modeled~\hl{Amar2016}. 

The boundary conditions for the energy equation includes a heated surface (\cref{eqn_thermal_bc_neumann}) and a constant-temperature surface (\cref{eqn_thermal_bc_dirichlet}). The boundary conditions for the pseudo-elasticity equation are a function of the surface temperature $T_q(x,t)$ for $x\in\Gamma_q$ using a B' table. The B' table....
\begin{equation}
    \bw(x,t) = \int_{0}^{t} \mathbf{v}(x,\tau)d\tau = \int_{0}^{t}\mathbf{f}\left(T_q(x,\tau)\right)d\tau
\end{equation}


\subsection{Full-Order Model: Discontinuous Galerkin Finite Element Method}

To obtain the full-order numerical solution, the governing equation is spatially discretized using variational principle of Discontinuous Galerkin (DG) to result in a high-dimensional system of ODEs. Note that the choice of DG approach here is mainly for theoretical convenience in the coarse-graining formulation. In Sec.~\hl{x} the full-order TPS ablation simulation is computed using standard FEM instead, and the equivalence between DG and standard FEM is noted upon their convergence.

\subsubsection{Numerical Solution}

To obtain the full-order numerical solution, the governing equation is spatially discretized using variational principles of Discontinuous Galerkin (DG) to result in a high-dimensional system of ordinary differential equations (ODEs). 

\subsubsection{Usage Within an Ablation Simulation}


\subsection{Formulation of Reduced-Physics Model}

In this section, a thermo-elastic RPM is derived to model the one-dimensional temperature distribution and surface recession for an ablating TPS. The RPM is composed of three coupled components: (1) a thermal solver based on FEM, (2) a pseudo-elastic solver for the FEM mesh, and (3) a lumped capacitance model (LCM) to model the volumetric energy source (i.e., the $\cQ(x,t)$ in \cref{eqn_thermal_pde}) interactions between the inter-connected components.

To start the RPM formulation, consider a partitioning of the general domain in Fig.~\ref{fig_general_domain} into smaller one-dimensional inter-connected material components as illustrated in Fig.~\hl{x}. Consider the $n$-th component with length $\ell^{(n)}$ illustrated in Fig.~\hl{x}, and for simplicity, assume that the material properties are independent of temperature.

For the $n$-th component, the left surface is exposed to the hypersonic flow (Neumann boundary condition), while the right surface is perfectly insulated (adiabatic boundary condition). The left surface is receding with a velocity $v(t)$ that is a function of the surface temperature $T_w(t)=T(0,t)$. This velocity is obtained from a B'-table lookup table,
\[
    v(t) = f(T_w(t))
\]
where $f(\cdot)$ is a cubic spline interpolation of B' and enthalpy tabulated data.

\begin{figure}[h]
    \centering
    \includegraphics[width=0.9\textwidth]{./figs/ablation.png}
    \label{fig_ablation_domain}
\end{figure}

The governing energy and elasticity equations are defined over the domain $\Omega=[0,\ell]$,
\begin{subequations}
    \begin{align}
        \rho c_p\left(\frac{\partial T}{\partial t} - v(x,t)\frac{\partial T}{\partial x}\right) - \frac{\partial}{\partial x}\left(k\frac{\partial T}{\partial x}\right) &= 0\label{eqn_thermal_1d}\\
        \frac{\partial}{\partial x}\left(\frac{\partial u}{\partial x}\right) &= 0\label{eqn_elasticity_1d}
    \end{align}\label{eqn_rpm}
\end{subequations}
with boundary conditions for the energy equation,
\begin{subequations}
    \begin{align}
        -k\frac{\partial T}{\partial x}\Bigg|_{x=0} &= q_b(t)\\
        -k\frac{\partial T}{\partial x}\Bigg|_{x=\ell} &= 0
    \end{align}
\end{subequations}
and for the elasticity equation,
\begin{subequations}
    \begin{align}
        u(0,t) &= \int_{t_0}^{t}v(\tau)d\tau = \int_{0}^{t} f(T_w(\tau))d\tau\\
        u(\ell,t) &= 0
    \end{align}
\end{subequations}

\subsection{Numerical Solution}

Along the one-dimensional domain, a numerical solution based on FEM is adopted for the energy equation, while an analytical solution is adopted for the pseudo-elastic mesh motion. The quasi-steady boundary conditions for the mesh motion are employed via a spline fit to the B' and enthalpy tabulated data as a function of surface temperature. The main results of the numerical approach are presented here and the reader is directed to Sec.~\ref{app_implementation} for details.

\subsubsection{Thermal Solver}

The result of the FEM discretization is a system of ODEs for the nodal temperature values given by,
\begin{equation}
    \mathbf{M}\frac{d\mathbf{T}}{dt} + \left(\mathbf{K} - \mathbf{C}(t)\right)\mathbf{T} = \mathbf{f}(t)
\end{equation}
where,
\begin{itemize}
    \item $\mathbf{M}$ is the mass matrix,
    \item $\mathbf{K}$ is the stiffness matrix,
    \item $\mathbf{C}(t)$ is the advection matrix,,
    \item $\mathbf{T}$ is the vector of nodal temperatures, and
    \item $\mathbf{f}(t)$ is the vector of external forces.
\end{itemize}

\subsubsection{Pseudo-Elastic Solver}

Note that \cref{eqn_elasticity_1d} is steady. Under the assumption that the mesh deformation is quasi-steady, it can be applied at each time step within an ablation simulation. For instance, for a known value of the wall temperature $T_w(t)$ the surface displacement 

Assuming the Young's modulus is constant, the PDE simplifies to,
\begin{equation}
    \frac{\partial^2 u}{\partial x^2} = 0
\end{equation}
which has the analytical solution,
\begin{equation}
    u(x,t) = a(t)x + b(t)
\end{equation}
Using the boundary conditions leads to,
\begin{equation}
    u(x,t) = u(0,t) * \left(1 - \frac{x}{\ell}\right)
\end{equation}
The mesh velocity is the time derivative of the displacement,
\begin{equation}
    v(x,t) = \frac{\partial u(x,t)}{\partial t} = v(t)\left(1 - \frac{x}{\ell}\right)
\end{equation}

\subsubsection{Coupling Scheme}

\subsubsection{Reduced-Physics Ablation Simulation}







