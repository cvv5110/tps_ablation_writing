\appendix

\section{Numerical Implementation}\label{app_implementation}

\subsection{Full-Order Model}

\subsection{Reduced-Physics Model}

This section outlines the derivation of the RPM for a generic TPS geometry. The RPM is formulated for $N$ interconnected one-dimensional components, and a three-component example is provided to demonstrate 

This section outlines the approach to derive the RPM for a series of $N$ interconnected one-dimensional components inside a TPS. Consider the splitting of the TPS into a series of one-dimensional strands, where the temperature distribution over the $n$-th strand is governed by,
\begin{equation}
    \rho c_p\frac{\partial T^{(n)}}{\partial t} - \rho c_p v^{(n)}(x,t)\frac{\partial T^{(n)}}{\partial x} - \frac{\partial}{\partial x}\left(k\frac{\partial T^{(n)}}{\partial x}\right) - \cQ^{(n)}_{\text{net}}(x,t) = 0
\end{equation}



where the temperature distribution of the $n$-th strand is related to the temperature distribution of the $(n-1)$-th and $(n+1)$-th strands via the volumetric energy source term,
\begin{equation}
    \cQ^{(n,n+1)}(x,t) = \frac{T^{(n)}(x,t) - T^{(n+1)}(x,t)}{R_{(n,)}}
\end{equation}
which is approximated using the \hl{x} relation from 

strand is coupled to 

each one-dimensional strand is coupled to the temperature distribution from its neighboring strand 

Let $\phi^{(e)}_i(x)$ with $i=1,2$ be two linear shape defined over the element $e_i=[x_{i},x_{i+1}]$ with length $h_e=x_{i+1} - x_i$,
\begin{equation}
    \phi^{(e)}_1(x) = \left\{\begin{matrix}
        \frac{x_{i+1} - x}{h_e},\quad x\in[x_i,x_{i+1}]\\
        0, \quad \text{otherwise}
    \end{matrix}\right.,\quad\phi^{(e)}_2(x) = \left\{\begin{matrix}
        \frac{x - x_i}{h_e},\quad x\in[x_i,x_{i+1}]\\
        0, \quad \text{otherwise}
    \end{matrix}\right.
\end{equation}
Letting,
\[
    x(\xi) = \frac{1-\xi}{2}x_i + \frac{1 + \xi}{2}x_{i+1}
\]
for $\xi\in[-1,1]$,
\begin{equation}
    \hat{\phi}_1^{(e)}(\xi) = \frac{1 - \xi}{2},\quad \hat{\phi}_2^{(e)}(\xi) = \frac{1 + \xi}{2}
\end{equation}

Multiply through by the test function,
\begin{equation}
    \int_{\Omega}\left[\rho c_p\frac{\partial T}{\partial t} - \rho c_p v(x,t)\frac{\partial T}{\partial x} - \frac{\partial}{\partial x}\left(k\frac{\partial T}{\partial x}\right) - \cQ(x,t)\right]\phi_i(x)dx = 0
\end{equation}

\begin{equation}
    \int_{\Omega}\rho c_p\frac{\partial T}{\partial t}\phi_i(x)dx - \int_{\Omega}\rho c_p v(x,t)\frac{\partial T}{\partial x}\phi_i(x)dx + \int_{\Omega} k\frac{\partial T}{\partial x}\frac{\partial\phi_i(x)}{\partial x} dx = k\frac{\partial T}{\partial x}\phi_i(x)\Bigg|_{\partial\Omega} + \int_{\Omega}\phi_j(x)\cQ(x,t)dx
\end{equation}
Perform the finite-element approximation,
\begin{equation}
    T(x,t)\approx\sum_jT_j(t)\phi_j(x)
\end{equation}
and define the matrix elements,
\begin{align}
    A_{ij} &= \int_{\Omega}\rho c_p\phi_i(x)\phi_j(x)dx\\
    C_{ij}(t) &= \int_{\Omega}\rho c_p v(x,t) \frac{\partial\phi_j}{\partial x}\phi_i(x)dx\\
    B_{ij} &= \int_{\Omega}k\frac{\partial\phi_i(x)}{\partial x}\frac{\partial\phi_j(x)}{\partial x}dx\\
    f_i(t) &= k\frac{\partial T}{\partial x}\phi_i(x)\Bigg|_{\partial\Omega} + \int_{\Omega}\phi_j(x)\cQ(x,t)dx
\end{align}
The time-dependent finite-dimensional ODE system for nodal temperatures $\mathbf{T}(t)$, including the ALE-induced advection effect from mesh motion, is given as,
\begin{equation}
    \mathbf{A}\frac{d\mathbf{T}}{dt} + \left(\mathbf{B} - \mathbf{C}(t)\right)\mathbf{T} = \mathbf{f}(t)
\end{equation}

The element-level expressions for the mass, stiffness, advection, and forcing vectors are given as,
\begin{align}
    M^{(e)}_{mn} &= \int_{x_i}^{x_{i+1}}\rho c_p\phi_m(x)\phi_n(x)dx = \rho c_p \frac{h_e}{6}\begin{pmatrix}
        2 & 1 \\ 1 & 2
    \end{pmatrix}\\
    K^{(e)}_{mn} &= \int_{x_i}^{x_{i+1}} k \frac{\partial \phi_m}{\partial x}\frac{\partial \phi_n}{\partial x} dx = \frac{k}{h_e}\begin{pmatrix}
        1 & -1 \\ -1 & 1
    \end{pmatrix}\\
    C^{(e)}_{mn}(t) &= \int_{x_i}^{x_{i+1}}\rho c_p v(x,t) \frac{\partial \phi_n(x)}{\partial x}\phi_m(x) dx\\
    f^{(e)}_1(t) &= \left(q(t),0\right)^T
\end{align}

\subsection{Three-Component Example}

\begin{subequations}
    \begin{align}
        \rho c_p \left(\frac{\partial T^{(1)}}{\partial t} - v^{(1)}(x,t)\frac{\partial T^{(1)}}{\partial x}\right) - \cQ^{(1)}_{\text{net}}(x,t) &= 0\\
        \rho c_p \left(\frac{\partial T^{(2)}}{\partial t} - v^{(2)}(x,t)\frac{\partial T^{(2)}}{\partial x}\right) - \cQ^{(2)}_{\text{net}}(x,t) &= 0\\
        \rho c_p \left(\frac{\partial T^{(3)}}{\partial t} - v^{(3)}(x,t)\frac{\partial T^{(3)}}{\partial x}\right) - \cQ^{(3)}_{\text{net}}(x,t) &= 0\\
    \end{align}
\end{subequations}
where,
\begin{subequations}
    \begin{align}
        \cQ^{(1)}_{\text{net}}(x,t) &=\\
        \cQ^{(2)}_{\text{net}}(x,t) &=\\
    \end{align}
\end{subequations}