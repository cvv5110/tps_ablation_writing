\appendix

\section{Numerical Implementation}\label{app_implementation}

\subsection{Full-Order Model}

Choosing appropriate basis functions $v$ and $w$ and using the Interior Penalty Galerkin (IPG) scheme ~\cite{Cohen and pernet 2018}, the variational bilinear form for \cref{eqn_thermal_pde} is,
\begin{equation}
    \sum_{i=1}^{M}\left(a^{1}_{\epsilon,i}(v,w) + a^{2}_{\epsilon,i}(v,w)\right) = \sum_{i=1}^{m}L_i(v)
\end{equation}
where $\epsilon$ is an user-specified parameter and,
\begin{subequations}
    \begin{align}
        a^{1}_{\epsilon,i}(v,w) &=\\
        a^{2}_{\epsilon,i}(v,w) &=\\
        L_i(v) &=
    \end{align}
\end{subequations}

\subsection{Reduced-Physics Model}

This section outlines the FEM implementation for the $N=3$ one-dimensional interconnected components used in Sec.~\hl{x}. Consider over the $i$-th element in Fig.~\hl{x} the linear basis,
\[
    \phi^{(i)}_1(x) = 1,\quad \phi^{(i)}_2(x) = \frac{2}{h^{(i)}}\left(x - x_c^{(i)}\right),\quad x\in\Omega^{(i)} = \left[x_0^{(i)},x_1^{(i)}\right]
\]
where $h^{(i)} = x_1^{(i)} - x_0^{(i)}$ and $x^{(i)}_c = (x^{(i)}_0 + x^{(i)}_1) / 2$. The 

The weak form of the 

Multiply through by the weight function $\phi_j(x)$ and integrate over the domain $\Omega^{(i)}$,
\begin{equation}
    \int_{\Omega^{(i)}}\left[\rho c_p\left(\frac{\partial T^{(i)}}{\partial t} - v^{(i)}(x,t)\frac{\partial T^{(i)}}{\partial x}\right) - \frac{\partial}{\partial x}\left(k\frac{\partial T^{(i)}}{\partial x}\right) - \cQ^{(i)}_{\text{net}}(x,t)\right]\phii_l(x)dx = 0
\end{equation}
Using integration by parts the natural boundary conditions are obtained,
\begin{align}
    \int_{\Omega^{(i)}}\rho c_p\phii_l(x)\frac{\partial\Ti}{\partial t}dx = -\int_{\Omega^{(i)}}k\frac{\partial\Ti}{\partial x}\frac{\partial\phii_l(x)}{\partial x} dx &+ \int_{\Omega^{(i)}}\rho c_p v(x,t)\phii_l(x)\frac{\partial\Ti}{\partial x} dx \notag\\
    & + k\frac{\partial\Ti}{\partial x}\phii_l(x)\Bigg|_{\partial\Omega} + \int_{\Omega^{(i)}}\phii_l(x)\cQnet^{(i)}(x,t)dx
\end{align}
Perform the finite-element approximation,
\begin{equation}
    T^{(i)}(x,t)\approx\sum_{k=1}^{n^{(i)}}\bar{u}^{(i)}_k(t)\phii_k(x)
\end{equation}
and define the matrix elements,
\begin{align}
    A^{(i)}_{kl} &= \int_{\Omega^{(i)}}\rho c_p\phii_k(x)\phii_l(x)dx\\
    B^{(i)}_{kl} &= -\int_{\Omega^{(i)}}k\frac{\partial\phii_k(x)}{\partial x}\frac{\partial\phii_l(x)}{\partial x}dx\\
    C^{(i)}_{kl}(t) &= \int_{\Omega^{(i)}}\rho c_p v^{(i)}(x,t)\phii_l(x)\frac{\partial\phii_k}{\partial x}dx\\
    f^{(i)}_l(t) &= k\frac{\partial T}{\partial x}\phii_l(x)\Bigg|_{\partial\Omega} + \int_{\Omega^{(i)}}\phii_l(x)\cQnet^{(i)}(x,t)dx
\end{align}
The semi-discrete form of the energy equation for the nodal temperatures $\bar{\vu}\in\mathbb{\bar{u}}$, including the ALE-induced advection effects from mesh motion, is given as,
\begin{equation}
    \mathbf{A}\dot{\bar{\vu}} = \left(\mathbf{B} + \mathbf{C}(t)\right)\bar{\vu} + \barf(t)
\end{equation}




The thermodynamic interaction between the components is modeled via the net volumetric energy source from \cref{eqn_net_volumetric_energy}, which for the three-components in Fig.~\hl{x} are described by,
\begin{subequations}
    \begin{align}
        \cQ^{(1)}_{\text{net}}(x,t) &= -\cQ^{(1,2)}(x,t)\\
        \cQ^{(2)}_{\text{net}}(x,t) &= \cQ^{(1,2)}(x,t) - \cQ^{(2,3)}(x,t)\\
        \cQ^{(3)}_{\text{net}}(x,t) &= \cQ^{(2,3)}(x,t)
    \end{align}
\end{subequations}


