\appendix

\section{Numerical Implementation}\label{app_implementation}

\subsection{Full-Order Model}

\subsection{Reduced-Physics Model}

This section outlines the DG-FEM implementation for the RPM with $N=3$ inter-connected one-dimensional components used in Sec.~\hl{x}. Consider the splitting of the TPS as in Fig.~\hl{x}. Over the $i$-th element with $x\in\Omega^{(i)}(t)$, consider the following linear basis set $\phi_k^{(i)}(x)$ with $k=1,2$ on the element $[x^{(i)}_{0}(t),x^{(i)}_{1}(t)]$ with length $h^{(i)}(t)=x^{(i)}_{1}(t) - x^{(i)}_{0}(t)$. For notational convenience, the time dependence of the spatial domain due to ablating surfaces is dropped. The orthogonal basis functions are defined as,
\begin{equation}
    \phi^{(i)}_1(x) = 1,\quad \phi^{(i)}_2(x) = \frac{2}{h^{(i)}}\left(x - x^{(i)}_c\right)
\end{equation}
where $x^{(i)}_c = (x^{(i)}_0 + x^{(i)}_1) / 2$. Let,
\[
    x(\xi) = \frac{1-\xi}{2}x^{(i)}_0 + \frac{1 + \xi}{2}x^{(i)}_{1}
\]
thus for $\xi\in[-1,1]$,
\begin{equation}
    \hat{\phi}^{(i)}_1(\xi) = 1, \quad \hat{\phi}^{(i)}_2(\xi) = \xi
\end{equation}

Multiply through by the weight function $\phi_j(x)$ and integrate over the domain $\Omega^{(i)}$,
\begin{equation}
    \int_{\Omega}\left[\rho c_p\left(\frac{\partial T^{(i)}}{\partial t} - v^{(i)}(x,t)\frac{\partial T^{(i)}}{\partial x}\right) - \frac{\partial}{\partial x}\left(k\frac{\partial T^{(i)}}{\partial x}\right) - \cQ^{(i)}_{\text{net}}(x,t)\right]\phii_l(x)dx = 0
\end{equation}
Using integration by parts the natural boundary conditions are obtained,
\begin{align}
    \int_{\Omega}\rho c_p\left(\frac{\partial\Ti}{\partial t}\phii_l(x) - v(x,t)\frac{\partial\Ti}{\partial x}\phii_l(x)\right)dx &+ \int_{\Omega} k\frac{\partial\Ti}{\partial x}\frac{\partial\phii_l(x)}{\partial x} dx\notag\\
    &=k\frac{\partial\Ti}{\partial x}\phii_l(x)\Bigg|_{\partial\Omega} + \int_{\Omega}\phii_l(x)\cQ(x,t)dx
\end{align}
Perform the finite-element approximation,
\begin{equation}
    T^{(i)}(x,t)\approx\sum_{k=1}^{n^{(i)}}\ui_k(t)\phii_k(x)
\end{equation}
and define the matrix elements,
\begin{align}
    A^{(i)}_{kl} &= \int_{\Omega}\rho c_p\phii_k(x)\phii_l(x)dx\\
    B^{(i)}_{kl} &= \int_{\Omega}k\frac{\partial\phii_k(x)}{\partial x}\frac{\partial\phii_l(x)}{\partial x}dx\\
    C^{(i)}_{kl}(t) &= \int_{\Omega}\rho c_p v^{(i)}(x,t)\phii_k(x)\frac{\partial\phii_l}{\partial x}dx\\
    f^{(i)}_l(t) &= k\frac{\partial T}{\partial x}\phii_l(x)\Bigg|_{\partial\Omega} + \int_{\Omega}\phii_l(x)\cQnet^{(i)}(x,t)dx
\end{align}
The time-dependent finite-dimensional ODE system for nodal temperatures $\mathbf{T}(t)$, including the ALE-induced advection effect from mesh motion, is given as,
\begin{equation}
    \mathbf{A}\frac{d\mathbf{T}}{dt} + \left(\mathbf{B} - \mathbf{C}(t)\right)\mathbf{T} = \mathbf{f}(t)
\end{equation}

On the element $(e)$ the expressions for mass, stiffness, advection, and forcing are given as,
\begin{align}
    M^{(i)}_{mn} &= \int_{x^{(i)}_i}^{x^{(i)}_{i+1}}\rho c_p\phi^{(i)}_m(x)\phi^{(i)}_n(x)dx = \rho c_p \frac{h^{(i)}}{6}\begin{pmatrix}
        2 & 1 \\ 1 & 2
    \end{pmatrix}\\
    K^{(i)}_{mn} &= \int_{x^{(i)}_i}^{x^{(i)}_{i+1}} k^{(i)} \frac{\partial \phi^{(i)}_m}{\partial x}\frac{\partial \phi^{(i)}_n}{\partial x} dx = \frac{k}{h_e}\begin{pmatrix}
        1 & -1 \\ -1 & 1
    \end{pmatrix}\\
    C^{(i)}_{mn}(t) &= \int_{x_i}^{x_{i+1}}\rho c_p v(x,t) \frac{\partial \phi_n(x)}{\partial x}\phi_m(x) dx\\
    f^{(i)}_1(t) &= \left(q(t),0\right)^T
\end{align}



The thermodynamic interaction between the components is modeled via the net volumetric energy source from \cref{eqn_net_volumetric_energy}, which for the three-components in Fig.~\hl{x} are described by,
\begin{subequations}
    \begin{align}
        \cQ^{(1)}_{\text{net}}(x,t) &= -\cQ^{(1,2)}(x,t)\\
        \cQ^{(2)}_{\text{net}}(x,t) &= \cQ^{(1,2)}(x,t) - \cQ^{(2,3)}(x,t)\\
        \cQ^{(3)}_{\text{net}}(x,t) &= \cQ^{(2,3)}(x,t)
    \end{align}
\end{subequations}


