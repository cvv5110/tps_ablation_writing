\appendix

\section{Mathematical Details}\label{app_implementation}

\subsection{Full-Order Model}

\subsubsection{Domain Discretization}

\subsubsection{Weak Form of Discontinuous Galerkin Method}

Choosing appropriate basis functions $\phi_k$ and $\phi_l$ and using the Interior Penalty Galerkin (IPG) scheme ~\cite{Cohen and pernet 2018}, the variational bilinear form for \cref{eqn_thermal_pde} is,
\begin{equation}
    \sum_{i=1}^{M}\left(a^{1}_{\epsilon,i}(\phi_k,\phi_l) + a^{2}_{\epsilon,i}(\phi_k,\phi_l)\right) = \sum_{i=1}^{m}L_i(\phi_k)
\end{equation}
where $\epsilon$ is an user-specified parameter and,
\begin{subequations}
    \begin{align}
        a^{1}_{\epsilon,i}(\phi_k,\phi_l) &= \int_{E^{(i)}}\left(\rho c_p \phik \frac{\partial \phil}{\partial t} + \nabla \phik\cdot\left(\mathbf{k}\nabla \phil\right) - \rho c_p \phik v\cdot\nabla\phil\right)dE^{(i)}\\
        a^{2}_{\epsilon,i}(\phik,\phil) &= -\sum_{j\in\cN_i\cup\dirichletset}\int_{\eij}\average{\bk\nabla\phik\cdot n}\jump{\phil}d\eij + \epsilon\sum_{j\in\cN_i\cup\dirichletset}\int_{\eij}\average{\bk\nabla\phil\cdot n}\jump{\phik}d\eij \notag\\
        &+ \sigma\sumneighbordirichlet\int_{\eij}\jump{\phik}\jump{\phil}d\eij\\
        L_i(v) &= \epsilon\sumneighbordirichlet\int_{\eij}\left(\bk\nabla\phil\cdot n\right)T_b d\eij + \int_{\eiq}\phik q_b d\eiq + \sigma\int_{\eiT}\phik T_b d\eiT
    \end{align}
\end{subequations}
In the bi-linear form above, the notations $\jump{}$ and $\average{}$ are respectively the jumps and averages at the boundary $\eij$ share by two elements $E_i$ and $E_j$,
\[
    \jump{u} = u\big|_{E_i} - u\big|_{E_j},\quad \average{u} = \frac{1}{2}\left(u\big|_{E_i} + u\big|_{E_j}\right), \quad \text{for } x\in\eij = E_i\cap E_j
\]
Furthermore, in the bi-linear form, the terms associated with $\sigma$ are introduced to enforce the Dirichlet boundary conditions; $\sigma$ is a penalty factor whose value can depend on the size of an element. Depending on the choice of $\epsilon$, the bi-linear form corresponds to symmetric IPG ($\epsilon=-1$), non-symmetric IPG ($\epsilon=1$), and incomplete IPG ($\epsilon=0$). All these schemes are consistent with the original PDE and have similar convergence rate with respect to mesh size. In the following derivations, the case $\epsilon=0$ is chosen for the sake of simplicity.

\subsubsection{Discontinuous Galerkin Model}

Next, the DG-based model is written in an element-wise form. For the $i$-th element, use a set of $P$ trial functions to represent the temperature as in \cref{eqn_element_temperature}. Without loss of generality, the trial functions are assumed to be orthogonal, so that $\int_{E^{(i)}} \phii_k(x) \phii_l(x) dx = \left|\Ei\right|\delta_{kl}$


Without loss of generality, the trial functions are assumed to be orthogonal, so that $\int_{E^{(i)}} \phii_k(x) \phii_l(x) dx = \left|\Ei\right|\delta_{kl}$, where $\left|\Ei\right|$ is the area $(n_d=2)$ or volume $(n_d=3)$ of the $i$-th element, and $\delta_{kl}$ is the Kronecker delta. 

Using test functions same as trial functions, the dynamics $\vu^{(i)}$ is obtained by evaluating the element-wise bi-linear forms,
\begin{equation}
    a^{1}(\phik^{(i)},T^{(i)})
\end{equation}

\subsection{Derivation of the Reduced-Physics Model}

This section outlines the derivation of the RPM. The 


DG-FEM implementation for $M$ components. Consider 




Multiply through by the weight function $\phi_j(x)$ and integrate over the domain $\Omega^{(i)}$,
\begin{equation}
    \int_{\Omega^{(i)}}\left[\rho c_p\left(\frac{\partial T^{(i)}}{\partial t} - v^{(i)}(x,t)\frac{\partial T^{(i)}}{\partial x}\right) - \frac{\partial}{\partial x}\left(k\frac{\partial T^{(i)}}{\partial x}\right) - \cQ^{(i)}_{\text{net}}(x,t)\right]\phii_l(x)dx = 0
\end{equation}
Using integration by parts the natural boundary conditions are obtained,
\begin{align}
    \int_{\Omega^{(i)}}\rho c_p\phii_l(x)\frac{\partial\Ti}{\partial t}dx = -\int_{\Omega^{(i)}}k\frac{\partial\Ti}{\partial x}\frac{\partial\phii_l(x)}{\partial x} dx &+ \int_{\Omega^{(i)}}\rho c_p v(x,t)\phii_l(x)\frac{\partial\Ti}{\partial x} dx \notag\\
    & + k\frac{\partial\Ti}{\partial x}\phii_l(x)\Bigg|_{\partial\Omega} + \int_{\Omega^{(i)}}\phii_l(x)\cQnet^{(i)}(x,t)dx
\end{align}
Perform the finite-element approximation,
\begin{equation}
    T^{(i)}(x,t)\approx\sum_{k=1}^{n^{(i)}}\bar{u}^{(i)}_k(t)\phii_k(x)
\end{equation}
and define the matrix elements,
\begin{align}
    A^{(i)}_{kl} &= \int_{\Omega^{(i)}}\rho c_p\phii_k(x)\phii_l(x)dx\\
    B^{(i)}_{kl} &= -\int_{\Omega^{(i)}}k\frac{\partial\phii_k(x)}{\partial x}\frac{\partial\phii_l(x)}{\partial x}dx\\
    C^{(i)}_{kl}(t) &= \int_{\Omega^{(i)}}\rho c_p v^{(i)}(x,t)\phii_l(x)\frac{\partial\phii_k}{\partial x}dx\\
    f^{(i)}_l(t) &= k\frac{\partial T}{\partial x}\phii_l(x)\Bigg|_{\partial\Omega} + \int_{\Omega^{(i)}}\phii_l(x)\cQnet^{(i)}(x,t)dx
\end{align}
The semi-discrete form of the energy equation for the nodal temperatures $\bar{\vu}\in\mathbb{\bar{u}}$, including the ALE-induced advection effects from mesh motion, is given as,
\begin{equation}
    \mathbf{A}\dot{\bar{\vu}} = \left(\mathbf{B} + \mathbf{C}(t)\right)\bar{\vu} + \barf(t)
\end{equation}




The thermodynamic interaction between the components is modeled via the net volumetric energy source from \cref{eqn_net_volumetric_energy}, which for the three-components in Fig.~\hl{x} are described by,
\begin{subequations}
    \begin{align}
        \cQ^{(1)}_{\text{net}}(x,t) &= -\cQ^{(1,2)}(x,t)\\
        \cQ^{(2)}_{\text{net}}(x,t) &= \cQ^{(1,2)}(x,t) - \cQ^{(2,3)}(x,t)\\
        \cQ^{(3)}_{\text{net}}(x,t) &= \cQ^{(2,3)}(x,t)
    \end{align}
\end{subequations}


