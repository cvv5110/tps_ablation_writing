\appendix

\section{Numerical Implementation}\label{app_implementation}

\subsection{Full-Order Model}

\subsection{Reduced-Physics Model}

This section outlines the DG-FEM implementation for the RPM with $N=3$ inter-connected one-dimensional components used in Sec.~\hl{x}. Consider the splitting of the TPS as in Fig.~\hl{x}. Over the element $e$ with $x\in\Omega^{(e)}(t)$, consider the following linear basis set $\phi_i^{(e)}(x)$ with $i=1,2$ on the element $[x^{(i)}_{0}(t),x^{(i)}_{1}(t)]$ with length $h^{(e)}(t)=x^{(e)}_{1}(t) - x^{(e)}_{0}(t)$. For notational convenience, the time dependence of the spatial domain due to ablating surfaces is dropped. The orthogonal basis functions are defined as,
\begin{equation}
    \phi^{(i)}_1(x) = 1,\quad \phi^{(i)}_2(x) = \frac{2}{h^{(i)}}\left(x - x^{(i)}_c\right)
\end{equation}
where $x^{(i)}_c = (x^{(i)}_0 + x^{(i)}_1) / 2$. Let,
\[
    x(\xi) = \frac{1-\xi}{2}x_i + \frac{1 + \xi}{2}x_{i+1}
\]
thus for $\xi\in[-1,1]$,
\begin{equation}
    \hat{\phi}^{(i)}_1(\xi) = 1, \quad \hat{\phi}^{(i)}_2(\xi) = \xi
\end{equation}

Multiply through by the weight function $\phi_j(x)$ and integrate over the domain $\Omega^{(i)}$,
\begin{equation}
    \int_{\Omega}\left[\rho c_p\left(\frac{\partial T^{(i)}}{\partial t} - v^{(i)}(x,t)\frac{\partial T^{(i)}}{\partial x}\right) - \frac{\partial}{\partial x}\left(k\frac{\partial T^{(i)}}{\partial x}\right) - \cQ^{(i)}_{\text{net}}(x,t)\right]\phi_j(x)dx = 0
\end{equation}
Using integration by parts the natural boundary conditions are obtained,
\begin{equation}
    \int_{\Omega}\rho c_p\frac{\partial T}{\partial t}\phi_j(x)dx - \int_{\Omega}\rho c_p v(x,t)\frac{\partial T}{\partial x}\phi_j(x)dx + \int_{\Omega} k\frac{\partial T}{\partial x}\frac{\partial\phi_j(x)}{\partial x} dx = k\frac{\partial T}{\partial x}\phi_j(x)\Bigg|_{\partial\Omega} + \int_{\Omega}\phi_j(x)\cQ(x,t)dx
\end{equation}
Perform the finite-element approximation,
\begin{equation}
    T^{(i)}(x,t)\approx\sum_{l=0}^{M-1} T_l(t)\phi_l(x)
\end{equation}
and define the matrix elements,
\begin{align}
    A_{ij} &= \int_{\Omega}\rho c_p\phi_j(x)\phi_j(x)dx\\
    C_{ij}(t) &= \int_{\Omega}\rho c_p v(x,t) \frac{\partial\phi_j}{\partial x}\phi_j(x)dx\\
    B_{ij} &= \int_{\Omega}k\frac{\partial\phi_j(x)}{\partial x}\frac{\partial\phi_j(x)}{\partial x}dx\\
    f_i(t) &= k\frac{\partial T}{\partial x}\phi_j(x)\Bigg|_{\partial\Omega} + \int_{\Omega}\phi_j(x)\cQ(x,t)dx
\end{align}
The time-dependent finite-dimensional ODE system for nodal temperatures $\mathbf{T}(t)$, including the ALE-induced advection effect from mesh motion, is given as,
\begin{equation}
    \mathbf{A}\frac{d\mathbf{T}}{dt} + \left(\mathbf{B} - \mathbf{C}(t)\right)\mathbf{T} = \mathbf{f}(t)
\end{equation}

On the element $(e)$ the expressions for mass, stiffness, advection, and forcing are given as,
\begin{align}
    M^{(e)}_{mn} &= \int_{x^{(e)}_i}^{x^{(e)}_{i+1}}\rho c_p\phi^{(e)}_m(x)\phi^{(e)}_n(x)dx = \rho c_p \frac{h^{(e)}}{6}\begin{pmatrix}
        2 & 1 \\ 1 & 2
    \end{pmatrix}\\
    K^{(e)}_{mn} &= \int_{x^{(e)}_i}^{x^{(e)}_{i+1}} k^{(e)} \frac{\partial \phi^{(e)}_m}{\partial x}\frac{\partial \phi^{(e)}_n}{\partial x} dx = \frac{k}{h_e}\begin{pmatrix}
        1 & -1 \\ -1 & 1
    \end{pmatrix}\\
    C^{(e)}_{mn}(t) &= \int_{x_i}^{x_{i+1}}\rho c_p v(x,t) \frac{\partial \phi_n(x)}{\partial x}\phi_m(x) dx\\
    f^{(e)}_1(t) &= \left(q(t),0\right)^T
\end{align}



The thermodynamic interaction between the components is modeled via the net volumetric energy source from \cref{eqn_net_volumetric_energy}, which for the three-components in Fig.~\hl{x} are described by,
\begin{subequations}
    \begin{align}
        \cQ^{(1)}_{\text{net}}(x,t) &= -\cQ^{(1,2)}(x,t)\\
        \cQ^{(2)}_{\text{net}}(x,t) &= \cQ^{(1,2)}(x,t) - \cQ^{(2,3)}(x,t)\\
        \cQ^{(3)}_{\text{net}}(x,t) &= \cQ^{(2,3)}(x,t)
    \end{align}
\end{subequations}
which from \cref{eqn_volumetric_energy_approximation} becomes,
\begin{subequations}
    \begin{align*}
        \cQ^{(1)}_{\text{net}}(x,t) &= -\frac{1}{R_{(1,2)}}\left(T^{(2)}(x,t) - T^{(1)}(x,t)\right)\\
        \cQ^{(2)}_{\text{net}}(x,t) &= \frac{1}{R_{(1,2)}}\left(T^{(2)}(x,t) - T^{(1)}(x,t)\right) - \frac{1}{R_{(2,3)}}\left(T^{(3)}(x,t) - T^{(2)}(x,t)\right)\\
        \cQ^{(3)}_{\text{net}}(x,t) &= \frac{1}{R_{(2,3)}}\left(T^{(3)}(x,t) - T^{(2)}(x,t)\right)
    \end{align*}
\end{subequations}

For example, for $x\in\Omega^{(2)}(t)$,
\begin{gather}
    \cQ^{(2)}_{\text{net}}(x,t) = \frac{1}{R_{(1,2)}}\left(T_1^{(2)}(t) + T_2^{(2)}(t)\xi^{(2)}(x,t)\right) - \frac{1}{R_{(2,3)}}\left(T_1^{(3)}(t) + T_2^{(3)}(t)\xi^{(3)}(x,t) - T_1^{(2)}(t) - T_2^{(2)}(t)\xi^{(2)}(x,t)\right)
\end{gather}
where,
\begin{equation}
    \xi^{(i)}(x,t) = \frac{2x(t) - (x_0^{(i)}(t) - x_1^{(i)}(t))}{x_1^{(i)}(t) - x_0^{(i)}(t)}
\end{equation}











