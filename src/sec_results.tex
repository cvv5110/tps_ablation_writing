\section{Application to Thermal Protection Systems}
In this section, the proposed PIROM approach is applied to the analysis of thermo-ablative multi-layered hypersonic TPS. The performance of the PIROM is evaluated in terms of \textit{accuracy}, \textit{generalizability}, and \textit{computational efficiency}, across a range of boundary condition and surface velocity model parametrizations. The results show PIROM to be a promising candidate for the solution of the impossible trinity of modeling.

\subsection{Problem Definition}
Consider the two-dimensional TPS configuration shown in Fig.~\hl{x} with constant material properties within each layer, dimensions, and BCs listed in Table~\hl{x}. Such configuration is representative of the TPS used for the initial concept 3.X vehicle in past studies~\cite{Klock2017}, and involves two main layers: an outer ablative layer, and an inner substrate layer. The top ablative layer may be composed of different materials, such as PICA or Avcoat, while the substrate layer is typically made of a high-temperature resistant material, such as carbon-carbon composite~\cite{Gasch2024}. The ablative layer, composed of $\tN=3$ ablative components, is subjected to strong time-varying and non-uniform heating, while the substrate layer, composed of one non-ablative component, is insulated adiabatically at the outer surface; the total number of components is thus $N=4$. 

The sources of non-linearities in the problem originate from the coupling between the thermodynamics and the temperature-dependent mesh motion, as well as the heterogeneities across material layers. As shown in Fig.~\hl{x}, perfect thermocouple devices are placed at the surfaces of the ablative layers for the collection of the high-fidelity temperature signals that are used in the following sections for training and testing the PIROM.

\subsection{Parametrization of Boundary Conditions and Surface Velocity Models}
The operating conditions of the TPS are specified by the boundary conditions, i.e., the heat flux, and the surface velocity model (SVM). Specifically, the heat flux on the Neumann BC is parametrized using $\xibc=\left\{\xi_0,\xi_1,\xi_2\right\}$, while the SVM is parametrized using $\xisvm=\{\alpha_1, \alpha_2,\alpha_3\}$. Thus, the heat flux and SVM over the $i$-th ablative component are expressed as,
\begin{equation}
    q(x,t;\xibc) = \xi_0e^{\xi_1 x}e^{\xi_2 t},\:\forall x\in\Gamma_{i,q},\quad \dot{w}_i(z_{u,i};\xisvm) = \alpha_i\left(z_{u,i} - u_{0,i}\right)
\end{equation}
where $\Gamma_{i,q}$, $z_{u,i}$, and $u_{0,i}$ correspond to the Neumann BC surface, the PIROM's surface temperature prediction, and the initial temperature of the $i$-th ablative component, respectively. The $\xi_0$ controls the magnitude of the heat flux, while $\xi_1$ and $\xi_2$ control the spatial and temporal variations, respectively. The constant $\alpha_i$ is a small material-dependent constant determined from the B' table, specifying the ablation velocity for a given change in surface temperature. 

\subsection{Data Generation}
Full-order solutions of the TPS are computed using the FEM multi-mechanics module of the \texttt{Aria} package~\cite{Clausen2024}, where the mesh is shown in Fig.~\hl{x}. The mesh consists of 2196 total elements, with 366 elements for each ablative component and 1098 elements for the substrate component. All solutions are computed for one minute from an uniform initial temperature of $T(x,t_0)=300$ K. Given an operating condition $\vxi=\left[\xibc,\xisvm\right]^\top$, a full-order solution consists of then collection of time-varying temperature and displacement fields $\left\{\left(t_k,\vu^{(l)}_{\text{HF}}(t_k),\vw^{(l)}_{\text{HF}}(t_k),\vxi^{(l)}\right)\right\}_{k=0}^{p-1}$, where $p$ is the number of time steps with a step size of $\Delta t \approx 10^{-3}$. The observable trajectories are representative of near-wall thermocouple sensing of hypersonic flows involving heat transfer. At each time instance $t_k$, a temperature reading is recorded from each ablative component using the thermocouples shown in Fig.~\hl{x}, resulting in three temperature signals, i.e., the observables $\vz_{text{HF}}\in\mathbb{R}^3$. Therefore, each full-order solution produces one trajectory of observables $\left\{\left(t_k,\vz^{(l)}_{\text{HF}}(t_k),\vxi^{(l)}\right)\right\}_{k=0}^{p-1}$. The goal of the PIROM is to predict the surface temperature and displacement as accurately as possible.

\subsubsection{Definition of Training and Testing Datasets}
The range of parameters used to generate the training and testing datasets are listed in Table~\hl{x}. The training and testing datasets are designed, respectively, to: (1) minimize the information that the PIROM can ``see'', and (2) to maximize the variability of test operating conditions to examine the PIROM's generalization performance. A total of $110$ normally-distributed data points for the BC parametrization are visualized in Fig.~\hl{x}, and the corresponding observable trajectories are shown in Fig.~\hl{x}. The training dataset $\cD_1$ includes $10$ trajectories with randomly selected BC parameters from the $110$ points, with nominal SVM parameters $\xisvm = \{1, 1, 1\}\times 10^{-6}$.

Two additional datasets are generated for testing. The dataset $\cD_2$ includes the remaining $100$ BC parameter values not considered in $\cD_1$, and the high-fidelity simulation are generated with the same nominal SVM parameters. The dataset $\cD_3$ includes $10$ SVM parameter perturbations


The testing dataset $\cD_2$ includes the remaining $100$ parameter values, and the high-fidelity simulations are generated with varying SVM parameters $\xisvm$ sampled from a uniform distribution within the ranges listed in Table~\hl{x}.

\subsection{Performance Metrics}

\subsection{Convergence Study}

\subsection{Generalization to Boundary Conditions}

\subsection{Generalization to Surface Velocity Models}

\subsection{Summary of Results}