\section{Physics-Infused Reduced-Order Modeling}\label{sec_pirom}
The formulation of PIROM for ablating TPS starts by connecting the FOM, i.e., the DG-FEM, and the RPM, i.e., the LCM, via a coarse-graining procedure. This procedure pinpoints the missing dynamics in the LCM when compared to DG-FEM. Subsequently, the Mori-Zwanzig (MZ) formalism is employed to determine the model form for the missing dynamics in PIROM. Lastly, the data-driven identification of the missing dynamics in PIROM is presented.

\subsection{Deriving the Reduced-Physics Model via Coarse-Graining}
The subsequent coarse-graining formulation is performed on the DG-FEM in \cref{eqn_full_dg} to derive the LCM in \cref{eqn_lcm}. This process constraints the trial function space of a full-order DG model to a subset of piece-wise constants, so that the variables $\vu$, matrices $\vA$, $\vB$, and $\vC$, and forcing vector $\vf$ are all approximated using a single state associated to the average temperature. Note that the coarse-graining is exclusively performed on the thermal dynamics, as it is the surface temperature that drives the one-dimensional recession via the SVM. Hence, the coarse-graining of the mesh dynamics is not included in the following procedure.

\subsubsection{Coarse-Graining of States}
Consider a DG model as in \cref{eqn_full_dg} for M elements and an LCM as in \cref{eqn_lcm} for $N$ components; clearly $M\gg N$. Let $\cV_j = \left\{i \big| E_i\in\Omega_j\right\}$ be the indices of the elements belonging to the $j$-th component, so $E_i\in\Omega_j$ for all $i\in \cV_j$. The number of elements in the $j$-th component is $|\cV_j|$. The average temperature on $\Omega_j$ is,
\begin{equation}
    \bar{u}_j = \frac{1}{|\Omega_j|}\sum_{i\in\cV_j}\intEi \vphi^{(i)}(x)^T\vu^{(i)} d\Omega = \frac{1}{\left|\Omega_j\right|}\sum_{i\in\cV_j} \left|E_i\right|\vvarphi_i^{j\top}\vu^{(i)},\quad j=1,2,\dots,N\label{eqn_average_temperature}
\end{equation}
where $\left|\Omega_j\right|$ and $\left|E_i\right|$ denote the area $(d=2)$ or volume $(d=3)$ of component $j$ and element $i$, respectively. The orthogonal basis functions are defined as $\vvarphi_i^{j\top} = \left[1,0,\dots,0\right]^{\top}\in\mathbb{R}^{P}$.

Conversely, given the average temperatures of the $N$ components, $\bvu$, the states of an arbitrary element $E_i$ is written as,
\begin{equation}
    \vu^{(i)} = \sum_{k=1}^{N}\vvarphi_i^{k}\bar{u}_k + \deltavu^{(i)}, \quad i=1,2,\dots,M\label{eqn_element_states}
\end{equation}
where $\vvarphi_{i}^{k}=0$ if $i\notin\cV_k$, and $\deltavu^{(i)}$ represents the deviation from the average temperature and satisfies the orthogonality condition $\vvarphi_{i}^{k\top}\deltavu^{(i)}=0$ for all $k$.

Equations \cref{eqn_average_temperature,eqn_element_states} are combined and written in matrix form as,
\begin{equation}
    \bvu = \vPhiplus\vu,\quad \vu = \vPhi\vu + \deltavu
\end{equation}
where $\vPhi\in\mathbb{R}^{MP\times N}$ is a matrix of $M\times N$ blocks, with the $(i,j)$-th block as $\vvarphiij$, $\vPhiplus\in\mathbb{R}^{N\times MP}$ is the left inverse of $\vPhi$, with the $(i,j)$-th block as $\vvarphi_{i}^{j+} = \frac{|E_i|}{|\Omega_j|}\vvarphiijT$, and $\deltavu$ is the collection of deviations. By their definitions, $\vPhiplus\vPhi = \vI$ and $\vPhiplus\deltavu = \vzero$.

\subsubsection{Coarse-Graining of Dynamics}
The dependence of the matrices with respect to the displacements $\vw$ is dropped to isolate the analysis based on coarsened variables. Consider a function of states in the form of $\vM\left(\vu\right)\vg(\vu)$, where $\vg:\mathbb{R}^{MP} \to \mathbb{R}^{MP}$ is a vector-valued function, and $\vM:\mathbb{R}^{MP} \to \mathbb{R}^{p\times MP}$ is a matrix-valued function with an arbitrary dimension $p$. Define the projection matrix $\vP=\vPhi\vPhiplus$ and the projection operator $\cP$ as,
\begin{align}
    \cP\left[\vM(\vu)\vg(\vu)\right] &= \vM\left(\vP{\vu}\right)\vg\left(\vP{\vu}\right)\notag\\
    &= \vM\left(\vPhi\bvu\right)\vg\left(\vPhi\bvu\right)\label{eqn_projection_operator}
\end{align}
so that the resulting function depends only on the average temperatures $\bvu$. Correspondingly, the residual operator $\cQ = \cI - \cP$, and $\cQ\left[\vM(\vu)\vg(\vu)\right] = \vM(\vu)\vg(\vu) - \vM\left(\vPhi\bvu\right)\vg\left(\vPhi\bvu\right)$. When the function is not separable, the projection operator is simply defined as $\cP\left[\vg(\vu)\right] = \vg\left(\vP\vu\right)$.

Subsequently, the operators defined above are applied to coarse-grain the dynamics. First, write the DG-FEM in \cref{eqn_full_dg} as,
\begin{equation}
    \dot{\vu} = \vA(\vu)^{-1}\vB(\vu)\vu + \vA(\vu)^{-1}\vC(\vu)\vu + \vA(\vu)^{-1}\vf(t)\label{eqn_full_dg_rearranged}
\end{equation}
and multiply both sides by $\vPhiplus$ to obtain,
\begin{equation}
    \vPhiplus\dot{\vu} = \vPhiplus\left(\vPhi\dot{\bvu} + \delta\dot{\vu}\right) = \dot{\bvu} = \vPhiplus\vr(\vu,t)
\end{equation}
Apply the projection operator $\cP$ and the residual operator $\cQ$ to the right-hand side to obtain,
\begin{equation}
    \dot{\bvu} = \cP\left[\vPhiplus\vr(\vu,t)\right] + \cQ\left[\vPhiplus\vr(\vu,t)\right]\equiv \vr^{(1)}(\vu,t) + \vr^{(2)}(\vu,t)\label{eqn_coarse_grained_dynamics}
\end{equation}
where $\vr^{(1)}(\vu,t)$ is resolved dynamics that depends on $\bvu$ only, and $\vr^{(2)}(\vu,t)$ is the un-resolved or residual dynamics. Detailed derivations and analysis of $\vr^{(1)}(\vu,t)$ and $\vr^{(2)}(\vu,t)$ can be found in the Appendix. 

It follows from Ref.~\cite{VargasVenegas2025} that the resolved dynamics is exactly the LCM, where the advection term reduces to zero, i.e., $\bvC(\bvu) = \vzero$ as shown in the Appendix. Using the notation from \cref{eqn_lcm}, it follows that,
\begin{align}
    \vr^{(1)}(\vu,t) &= \bvA(\bvu)^{-1}\bvB(\bvu)\bvu + \bvA(\bvu)^{-1}\bvC(\bvu)\bvu + \bvA(\bvu)^{-1}\bvf(\bvu)\notag\\
    &= \bvA(\bvu)^{-1}\bvB(\bvu)\bvu + \bvA(\bvu)^{-1}\bvf(t)\label{eqn_resolved_dynamics_no_advection}
\end{align}
where the following relations hold,
\begin{subequations}
    \begin{align}
        \bvA(\bvu) &= \vW\left(\vPhiplus\vA\left(\vPhi\bvu\right)^{-1}\vPhi\right)^{-1} &\quad \bvC(\bvu) &= \vzero \\
        \bvB(\bvu) &= \vW\vPhiplus\vB\left(\vPhi\bvu\right)\vPhi &\quad \bvf(t) &= \vW\vPhiplus\vf
    \end{align}\label{eqn_coarse_grained_matrices}
\end{subequations}
where $\vW\in\mathbb{R}^{N\times N}$ is a diagonal matrix with the $i$-th element as $\left[\vW\right]_{ii} = \left|\cV_k\right|$ if $i\in\cV_k$. The examination of the second residual term $\vr^{(2)}(\vu,t)$ in \cref{eqn_coarse_grained_dynamics} is shown in the Appendix, and demonstrates that the physical sources of missing dynamics in the LCM include: the approximation of non-uniform temperature within each component as a constant, and the elimination of the advection term due to coarse-graining. In sum, the above results not only show that the LCM is a result of coarse-graining of the full-order DG-FEM, but also reveal the discrepancies between the LCM and the DG-FEM. These discrepancies propagate into the SVM, which as a result of the averaging in the LCM formulation, under-predicts the surface recession rates. In the subsequent section, the discrepancies in the LCM are corrected to formulate the PIROM.

\subsection{Physics-Infusion Via Mori-Zwanzig Formalism}
The Mori-Zwanzig (MZ) formalism is an operator-projection technique used to derive ROMs for high-dimensional dynamical systems, especially in statistical mechanics and fluid dynamics~\cite{Parish2017a,Parish2017,Duraisamy2025}. It provides an exact reformulation of a high-dimensional Markovian dynamical system, into a low-dimensional observable non-Markovian dynamical system. The proposed ROM is subsequently developed based on the approximation to the non-Markovian term in the observable dynamics. Particularly, \cref{eqn_coarse_grained_dynamics} shows that the DG-FEM dynamics can be decomposed into the resolved dynamics $\vr^{(1)}(\vu,t)$ and the orthogonal dynamics $\vr^{(2)}(\vu,t)$, in the sense of $\cP\vr^{(2)}=0$. In this case, the MZ formalism can be invoked to express the dynamics $\bvu$ in terms of $\bvu$ alone as the projected Generalized Langevin Equation (GLE)~\cite{Parish2017a,Parish2017,Duraisamy2025},
\begin{equation}
    \dot{\bvu}(t) = \vr^{(1)}(\bvu,t) + \int_{0}^{t}\tvkappa(t,s,\bvu)ds\label{eqn_gle}
\end{equation}
where the first and second terms are referred to as the Markovian and non-Markovian terms, respectively. The non-Markovian term accounts for the effects of past un-resolved states on the current resolved states via a memory kernel $\tvkappa(t,s,\bvu)$, which in practice is computationally expensive to evaluate.

\subsubsection{Markovian Reformulation}
This section details the formal derivation of the PIROM as a system of ODEs for the thermal dynamics, based on approximations to the memory kernel. Specifically, the kernel $\tvk$ is examined via a leading-order expansion, based on prior work~\cite{Yu2024coarse}; this can be viewed as an analog of zeroth-order holding in linear system theory with a sufficiently small time step. In this case, the memory kernel is approximated as,
\begin{equation}
    \tvkappa(t,s,\bvu) \approx \vr^{(1)}(\bvu,t) \cdot \nabla_{\bvu}\vr^{(2)}\left(\vPhi\bvu,t\right)
\end{equation}
Note that the terms in $\vr^{(1)}$ have a common factor $\bvA^{-1}$; this motivates the following heuristic modification of the model form in \cref{eqn_gle},
\begin{subequations}
    \begin{align}
        \dot{\bvu} &= \vr^{(1)}(\bvu,t) + \bvA^{-1}(\bvu)\int_{0}^{t}\vkappa(t,s,\bvu)ds\label{eqn_pirom_gle}\\
        \bvA(\bvu)\dot{\bvu} &= \bvB\left(\bvu\right)\bvu + \bvf(t) + \int_{0}^{t}\vkappa(t,s,\bvu)ds\label{eqn_lcm_gle}
    \end{align}
\end{subequations}
where the original kernel $\tvkappa$ is effectively normalized by $\bvA^{-1}$. Intuitively, such choice of kernel reduces its dependency on the averaged material properties, and simplifies the subsequent design of model form.

Subsequently, the hidden states are introduced to ``Markovianize'' the system \cref{eqn_gle}. In this manner, \cref{eqn_lcm_gle} is converted into a pure state-space model, with the functional form of the LCM retained; since LCM is a physics-based model, then it encodes the physical information and retains explicit parametric dependence of the problem. Consider the representation of the kernel as a finite sum of simpler functions, e.g., exponentials,
\begin{equation}
    \vkappa(t,s,\bvu) = \sum_{j=1}^{m}\cK_j(t,s,\bvu)\left[\vp_j + \vd_j(\bvu)\right]\phi_j(s,\bvu)
\end{equation}
where,
\begin{equation}
    \cK_j(t,s,\bvu) = e^{-\int_{s}^{t}\left(\lambda_j + e_j(\bvu)\right)d\tau},\quad \phi_j(s,\bvu) = \vq_j^{\top}\bvu(s) + \vg_j(\bvu)^{\top}\bvu(s) + \vr_j^{\top}\bvf(s)
\end{equation}
with suitable coefficients $\vp_j,\vd_j,\vq_j,\vg_j,\vr_j\in\mathbb{R}^{N}$ and decay rates $\lambda_j,e_j(\bvu)>0$, that need to be identified from data. 

Define the hidden states as,
\begin{equation}
    \beta_j(t) = \int_{0}^{t}\cK_j(s,\bvu)\phi_j(s,\bvu)ds
\end{equation}
then through its differentiation with respect to time,
\begin{equation}
    \dot{\beta}_j(t) = -\left[\lambda_j + e_j(\bvu)\right]\beta_j(t) + \vq_j^{\top}\bvu(t) + \vg_j(\bvu)^{\top}\bvu(t) + \vr_j^{\top}\bvf(t)
\end{equation}
and the memory term becomes,
\begin{equation}
    \int_{0}^{t}\vkappa(t,s,\bvu)ds = \sum_{j=1}^{m}\left[\vp_j + \vd_j(\bvu)\right]\beta_j(t)
\end{equation}
Then, \cref{eqn_lcm_gle} is recast as the extended Markovian system,
\begin{subequations}
    \begin{align}
        \bvA(\bvu)\dot{\bvu} &= \bvB(\bvu)\bvu + \left[\vP + \vD(\bvu)\right]\vbeta + \bvf(t)\\
        \dot{\vbeta} &= \left[\vQ + \vG(\bvu)\right]\bvu + \left[\vE(\bvu) - \vLambda\right]\vbeta + \vR\bvf(t)
    \end{align}\label{eqn_extended_markovian_system}
\end{subequations}
where the data-driven operators associated to the hidden dynamics are collected as,
\begin{subequations}
    \begin{align}
        \vLambda &= \text{diag}\left(\lambda_1,\lambda_2,\dots,\lambda_m\right)\in\mathbb{R}^{m\times m},\quad & \vP &= \left[\vp_1,\vp_2,\dots,\vp_m\right]\in\mathbb{R}^{N\times m}\\
        \vD(\bvu) &= \left[\vd_1(\bvu),\vd_2(\bvu),\dots,\vd_m(\bvu)\right]\in\mathbb{R}^{N\times m},\quad & \vQ &= \left[\vq_1,\vq_2,\dots,\vq_m\right]\in\mathbb{R}^{m\times N}\\
        \vG(\bvu) &= \left[\vg_1(\bvu),\vg_2(\bvu),\dots,\vg_m(\bvu)\right]\in\mathbb{R}^{m\times N},\quad & \vR &= \left[\vr_1,\vr_2,\dots,\vr_m\right]\in\mathbb{R}^{m\times N}\\
        \vE(\bvu) &= \text{diag}\left(e_1(\bvu),e_2(\bvu),\dots,e_m(\bvu)\right)\in\mathbb{R}^{m\times m}
    \end{align}
\end{subequations}
The form of the temperature-dependent matrices $\vD(\bvu)$, $\vG(\bvu)$, and $\vE(\bvu)$ is provided in the next section. Note that since the hidden states $\vbeta$ serve as the memory, their initial conditions are set to zero, i.e., $\vbeta(t_0) = \vzero$, no memory at the beginning. The physics-infused model in \cref{eqn_extended_markovian_system} retains the structure of the LCM, while the hidden states account for missing physics through corrections to the stiffness and advection matrices, as well as the forcing term.

\subsubsection{Coupled Physics-Infused Model}
The next step involves coupling the physics-infused model in \cref{eqn_extended_markovian_system} with the SVM in \cref{eqn_svm} to form the PIROM for ablating TPS. To this end, define the observables as the surface temperature $\vz_u\in\mathbb{R}^{\tN}$ and displacements $\vz_w\in\mathbb{R}^{\tN}$ for $\tN\leq N$ ablating components to define the observable vector as $\vz=\left[\vz_u,\vz_w\right]^\top\in\mathbb{R}^{n_z}$ with $n_z=2\tN$ as the total number of observables.

Collect the RPM and hidden states into a single state vector $\vy = \left[\bvu,\vw,\vbeta\right]^\top\in\mathbb{R}^{n_y}$, where $n_y = N + \tN + m$, and define a data-driven operator $\vM\in\mathbb{R}^{n_z\times n_{y}}$ to define the PIROM's observable as,
\begin{equation}
    \vz = \vM\vy\label{eqn_pirom_observable_mapping}
\end{equation}
where,
\begin{equation}
    \vM = \begin{bmatrix}
        \vM_{\bu} & \vzero & \vM_{\beta}\\
        \vzero & \vI & \vzero
    \end{bmatrix}
\end{equation}
includes the matrices $\vM_{\bu}\in\mathbb{R}^{\tN\times N}$ and $\vM_{\beta}\in\mathbb{R}^{\tN\times m}$, which computes the surface temperature observable from the RPM states and hidden states, respectively. The PIROM is coupled to the SVM in \cref{eqn_svm} by leveraging \cref{eqn_pirom_observable_mapping} to compute the surface recession velocity. Thus, the PIROM is formally stated as,
\begin{subequations}
    \begin{align}
        \vcA\dot{\vy} &= \left[\vcB + \vcC\right]\vy + \vcF(t)\\
        \vz &= \vM\vy
    \end{align}\label{eqn_pirom_general}
\end{subequations}
where,
\begin{subequations}
    \begin{gather}
        \vcA = \begin{bmatrix}
            \bvA & \vO & \vO\\
            \vO & \vI & \vO\\
            \vO & \vO & \vI
        \end{bmatrix}\in\mathbb{R}^{n_y\times n_y},\quad \vcB = \begin{bmatrix}
            \bvB & \vO & \vP\\
            \vXi\vM_u & \vO & \vXi\vM_{\beta}\\
            \vQ & \vO & -\vLambda
        \end{bmatrix}\in\mathbb{R}^{n_y\times n_y},\\
        \vcC = \begin{bmatrix}
            \vO & \vO & \vD(\bvu)\\
            \vO & \vO & \vO\\
            \vG(\bvu) & \vO & \vE(\bvu)
        \end{bmatrix}\in\mathbb{R}^{n_y\times n_y},\quad
        \vcF = \begin{bmatrix}
            \bvf(t) \\
            -\tvf\\
            \vR\bvf(t)
        \end{bmatrix}\in\mathbb{R}^{n_y}
    \end{gather}
\end{subequations}

The learnable parameters in the PIROM are collected as,
\begin{equation}
    \vTheta = \left\{\vP,\vQ,\vR,\vD(\bvu),\vG(\bvu),\vE(\bvu),\vM_{u},\vM_{\beta}\right\},\in\mathbb{R}^{n_{\theta}}
\end{equation}
Particularly, the matrices $\vP,\vLambda,\vQ,\vR$ are constants that need to be identified from data, and account for the effects of coarse-graining on the stiffness and forcing matrices. The matrices $\vD(\bvu),\vE(\bvu),\vG(\bvu)$ are state-dependent matrices, and account for the effects of coarse-graining on the advection matrix due to mesh motion. Leveraging the DG-FEM formula for the advection matrix in \cref{eqn_advection_matrix_element_kl} in the Appendix, and noting that the ablating velocity in \cref{eqn_boundary_velocity} imposes the boundary condition for the mesh motion, the state-dependent matrices for the $i$-th component are written as,
\begin{equation}
    \vD(\bvu) \approx \dot{\vw}(\bvu)\odot_{\text{r}}\vD,\quad \vG(\bvu) \approx \vG\odot_{\text{r}}\dot{\vw}(\bvu), \quad \vE(\bvu) \approx \dot{\vW}(\bvu)\odot\vE\label{eqn_advection_matrix_corrections}
\end{equation}
where $\dot{\vw}(\bvu)$ is the SVM based on the observable temperature $\bvu$, $\odot_{\text{r}}$ is the row-wise multiplication, and $\dot{\vW}$ is the concatenation of $\dot{\vw}$ for $\tilde{m}$ times, where $\tilde{m}$ corresponds to the number of hidden states per component, i.e., $m = N\tilde{m}$.

The PIROM in \cref{eqn_pirom_general} incorporates explicit information on the material properties, boundary conditions, and surface recession, and is designed to generalize across parametric variations in these inputs. Moreover, the hidden dynamics in \cref{eqn_extended_markovian_system} are interpretable, as these retain the functional form of the DG-FEM in \cref{eqn_full_dg}. The next step is focused on identifying the unknown data-driven parameters $\vTheta$ characterizing the hidden dynamics.

\subsection{Learning the Hidden Dynamics}
Learning of the PIROM is achieved through a gradient-based neural-ODE-like approach~\cite{Chen2019}. For ease of presentation, consider the compact form of the PIROM in \cref{eqn_pirom_general},
\begin{equation}
    \vcD\left(\dot{\vy},\vy,\vxi,\vcF;\vTheta\right) = \vzero\label{eqn_pirom_compact}
\end{equation}
where $\vxi$ defines the model parameters, i.e., material properties and B' tables, while $\vcF$ represents the forcing terms, i.e., the boundary conditions.

Consider a dataset of $N_s$ high-fidelity \textit{surface temperature} observable trajectories $\vz_{\text{HF}}$, sampled at $p$ time instances $\{t_k\}_{k=0}^{p-1}$, for different parameter settings $\{\vxi^{(l)}\}_{l=1}^{N_s}$ and forcing functions $\{\vcF^{(l)}(t)\}_{l=1}^{N_s}$. The dataset is expressed as,
\begin{equation}
    \cD = \left\{\left(t_k,\vz^{(l)}_{\text{HF}}(t_k),\vxi^{(l)},\vcF^{(l)}(t_k)\right)\right\}_{l=1}^{N_s},\quad k=0,1,\dots,p-1
\end{equation}
In this work, the dataset contains only surface temperature observables -- all high-fidelity information regarding the surface displacements \textit{are assumed to be unavailable during learning}.

The learning problem is formulated as the following differentially-constrained problem,
\begin{subequations}
    \begin{align}
        \min_{\vTheta} &\quad\cJ\left(\vTheta;\cD\right) = \sum_{l=1}^{N_s}\int_{t_0}^{t_f}\ell\left(\vz_u^{(l)},\vz_{\text{HF}}^{(l)}\right)dt\\
        \text{s.t.}\quad \vzero &= \vcD\left(\dot{\vy}^{(l)},\vy^{(l)},\vxi^{(l)},\vcF^{(l)};\vTheta\right)
    \end{align}
\end{subequations}
for $l=1,2,\dots,N_s$, the objective is to minimize the discrepancy between the high-fidelity and PIROM predictions for the $l$-th trajectory with $\ell\left(\vz_u^{(l)},\vz_{\text{HF}}^{(l)}\right) = \left\|\vz_u^{(l)} - \vz^{(l)}_{\text{HF}}\right\|_2^2$.

The gradient-based optimization loop is based on the adjoint variable $\vlambda$, governed by the adjoint differential equation,
\begin{subequations}
    \begin{align}
        \frac{\partial\ell}{\partial\vy} + \vlambda^\top\frac{\partial\vcD}{\partial\vy} - \frac{d}{dt}\left(\vlambda^\top\frac{\partial\vcD}{\partial\dot{\vy}}\right) &= \vzero\\
        \vlambda(t_f)^\top\frac{\partial\vcD}{\partial\dot{\vy}(t_f)} &=\vzero
    \end{align}\label{eqn_adjoint_equation}
\end{subequations}
Once $\vlambda$ is solved, the gradient is computed as,
\begin{equation}
    \nabla_{\vTheta}\cJ = \frac{1}{N_s}\sum_{l=1}^{N_s}\int_{t_0}^{t_f}\left(\frac{\partial\ell}{\partial\vTheta} + \left(\vlambda^{(l)}\right)^\top\frac{\partial\vcD}{\partial\vTheta}\right)dt
\end{equation}

\hl{Discussion on TSA?}
